\documentclass{article}
\usepackage[utf8]{inputenc}
\usepackage{amsfonts}
\usepackage{amsmath}
\usepackage{amssymb}
\usepackage{framed}
\usepackage[framemethod=tikz]{mdframed}
\usepackage{braket}

\newcommand{\Z}{\mathbb{Z}}
\newcommand{\R}{\mathbb{R}}
\newcommand{\C}{\mathbb{C}}
\newcommand{\Q}{\mathbb{Q}}
\newcommand{\N}{\mathbb{N}}

\newcommand{\qed}{\null\nobreak\hfill\ensuremath{\square}}

\definecolor{mycolor}{rgb}{0.122, 0.435, 0.698}
\definecolor{exampleBorder}{rgb}{0.8, 0.58, 0.46}
\definecolor{propBorder}{rgb}{0.0, 0.5, 0.0}
\definecolor{theoremBorder}{rgb}{0.2, 0.2, 0.6}
\definecolor{proofBorder}{rgb}{1.0, 0.44, 0.37}
\definecolor{rmkBorder}{rgb}{0.75, 0.58, 0.89}
\definecolor{claimBorder}{rgb}{0.0, 0.42, 0.24}
\definecolor{lemmaBorder}{rgb}{0.4, 0.6, 0.8}

\newmdenv[nobreak=true,innerlinewidth=0.5pt, roundcorner=4pt,linecolor=mycolor,innerleftmargin=12pt,innerrightmargin=12pt,innertopmargin=12pt,innerbottommargin=12pt]{definition}
\newmdenv[leftmargin=1pt, rightmargin=1pt, nobreak=true,innerlinewidth=0.5pt, roundcorner=4pt,linecolor=exampleBorder,innerleftmargin=12pt,innerrightmargin=12pt,innertopmargin=12pt,innerbottommargin=12pt]{example}
\newmdenv[nobreak=true,innerlinewidth=0.5pt, roundcorner=4pt,linecolor=propBorder,innerleftmargin=12pt,innerrightmargin=12pt,innertopmargin=12pt,innerbottommargin=12pt]{prop}
\newmdenv[nobreak=true,innerlinewidth=0.5pt, roundcorner=4pt,linecolor=theoremBorder,innerleftmargin=12pt,innerrightmargin=12pt,innertopmargin=12pt,innerbottommargin=12pt]{theorem}
\newmdenv[nobreak=true,innerlinewidth=0.5pt, roundcorner=4pt,linecolor=proofBorder,innerleftmargin=12pt,innerrightmargin=12pt,innertopmargin=12pt,innerbottommargin=12pt]{proof}
\newmdenv[nobreak=true,innerlinewidth=0.5pt, roundcorner=4pt,linecolor=rmkBorder,innerleftmargin=12pt,innerrightmargin=12pt,innertopmargin=12pt,innerbottommargin=12pt]{remark}
\newmdenv[nobreak=true,innerlinewidth=0.5pt, roundcorner=4pt,linecolor=claimBorder,innerleftmargin=12pt,innerrightmargin=12pt,innertopmargin=12pt,innerbottommargin=12pt]{claim}
\newmdenv[nobreak=true,innerlinewidth=0.5pt, roundcorner=4pt,linecolor=lemmaBorder,innerleftmargin=12pt,innerrightmargin=12pt,innertopmargin=12pt,innerbottommargin=12pt]{lemma}

\title{Math 320: Chapter 1}
\author{rctcwyvrn}
\date{September 2020}

\begin{document}

\maketitle

\section{Common sets}
\begin{itemize}
	\item Natural numbers $\N$ (Rudin uses $\mathbb{J}$)
		\begin{itemize}
			\item Set notation $\left\{ 1, 2, 3, 4, \ldots\right\}$
			\item Closed under addition and multiplication
			\item Not closed under subtraction (might get a negative number)
		\end{itemize}
	\item Integers $\Z$
		\begin{itemize}
			\item Set notation $\ldots -3, -2, -1, 0,1,2,3,4,\ldots$
			\item Closed under subtraction and addition
			\item But not under division
		\end{itemize}
	\item Rationals $\Q$
		\begin{itemize}
			\item Set notation: $\left\{\frac{m}{n}: m\in\Z, n\in \N \right\}$
			\begin{itemize}
				\item Where $\frac{m_1}{n_1} = \frac{m_2}{n_2}$ iff $m_1*n_2 = m_2*n_1$
				\item Our intuitive idea of division
			\end{itemize}
		\item Alternative we could define $\Q$ in a more straightforward way
			\begin{itemize}
				\item Define $\Q$ as the set of ordered pairs $\left\{(m,n): m\in\Z, n\in \N  \right\}$
				\item We also need to make sure the equivalent fractions are accounted for
				\item where $(m_1,n_1)$ is equivalent to $(m_2,n_2)$ (ie $(m_1,n_1) \sim (m_2,n_2)$) if $m_1n_2 = m_2n_1$
			\end{itemize}
		\item Closed under addition, subtraction, multiplication, and division (if divisor is non-zero)
		\item Question: Are the rationals sufficient for everything we want to do in real analysis (in calculus)?
		\begin{itemize}
			\item Can we do things in calculus? 
			\item Can we take limits? (and have them work properly?)
		\end{itemize}
		\item Nope! (note: the name of this course is \textbf{real} analysis, not \textbf{rational} analysis
		\item The problem is that the rationals have holes, they're not "filled in all the way" like the reals are
	\end{itemize}
\end{itemize}
\begin{example} 
		\textbf{Example:} Holes in the rationals \\
		~\\
		(Rudin 1.1a)  We want to show that there's a number we can't reach in the rationals ($\sqrt{2}$). $\nexists p\in\Q : p^2=2$  
		\begin{proof} 
		\textbf{Proof:} \\
		~\\
		By contradiction: Suppose that $\exists p\in\Q : p^2=2$. Then since $p$ is rational, we can write $p = \frac{m}{n}$, $m\in\Z$, $n\in\N$.
		\begin{itemize}
			\item WLOG we may suppose that $m$ and $n$ are not both even (because if not then we could just divide both by 2 and have a new $m$ and $n$ and repeat until this statement is true)
			\begin{itemize}
				\item Q: How can we be sure that this dividing by 2 will eventually end? (More explicitly, how many times do we need to do it?)
			\end{itemize}
			\item We have $2 = p^2 = \frac{m^2}{n^2}$
			\begin{align}
				2 &= \frac{m^2}{n^2} \\
				m^2 &= 2n^2
			\end{align}
		\item So $m$ must be even
		\item Ie $m = 2k$ for some odd integer $k$
		\begin{align}
			2n^2 &= m^2 \\
			2n^2 &= (2k)^2 \\
			n^2 &= 2k^2 
		\end{align}
		\item So $n$ must be even, but we said that one of them had to be odd.
		\item Contradiction!
		\end{itemize}
		\end{proof}
	\end{example}
	\begin{example} \textbf{Example:} \\
		\\
		(Rudin 1.1b) 
		\begin{itemize}
			\item Let $A = \left\{ p\in\Q: p>0, p^2<2\right\}$
			\item Let $B = \left\{ p\in\Q : p>0, p^2>2\right\}$
			\item Consider the area where the two sets meet. We know that they meet at $\sqrt{2}$, which is not a rational. 
			\item So we know that the set of rationals has little holes in it, like $\sqrt{2}$ which make us unable to use limits, which are important for real analysis 
		\end{itemize}
		Then:
		\begin{itemize}
			\item $\forall p\in A, \exists q\in A: p<q$ (A does not have a largest element)
			\item Similarly $B$ does not have a smallest element ($\forall p\in B, \exists q\in B: q<p$
		\end{itemize}
		\begin{proof} 
		\textbf{Proof:} \\
		~\\
		First one
		\begin{itemize}
			\item Consider arbitrary $p\in A$. Try $q = \frac{2p+2}{2+p}$ 
			\item Check $q\in\Q$, the denominator is non-zero and rational, so the result is rational.
			\item Check $q>0$, both numerator and denominator are positive
			\item Check $q^2<2$.
			\begin{align}
				q^2 &= \frac{(2p+2)^2}{(2+p)^2} \\
				    &= \frac{2(p^2-2)}{(p+2)^2} +2
			\end{align}
			\item The fraction is less than zero, because the numerator is negative ($p^2 <2$ because $p\in A$), so $q^2<2)$
			\item Check that $p<q$. Well $q = p + \frac{2-p^2}{2+p}$, which is positive, so $q>p$
		\end{itemize}
		Second one
		\begin{itemize}
			\item Exercise (Try the exact same choice of $q$)
		\end{itemize}
		\end{proof}
	\item Q: Where did that $q$ come from? (Think calculus)
	\item Q: Why do we care so much that there is $q^2=2$, but don't care that there isn't a $q^2 = -1$?
\end{example}
\subsection{Ordered Sets}
\begin{definition} 
\textbf{Definition:} Set order \\
~\\
An {\color{blue} \textbf{order ($<$)}} on a set $S$ is a relation st
\begin{itemize}
	\item Every pair of distinct elements $x,y\in S$, exactly one of $x<y$, $x>y$, $x=y$ is true
	\item Transitivity: For $x,y,z\in S$, $x<y$ $y<z$ implies $x<z$
\end{itemize}
\end{definition}
\begin{definition} 
\textbf{Definition:} Ordered set \\
~\\
A pair $(S,<)$ of an order and a set (duh)
\begin{itemize}
	\item Notation: Just $S$ if the order is obvious from the context
\end{itemize}
\end{definition}
\begin{example} 
\textbf{Example:} Ordered sets \\
~\\
With ordering $x<y$ if $y-x$ is positive
\begin{itemize}
	\item $S = \N$
	\item $S = \Z$
	\item $\Q$, $\R$ etc
\end{itemize}
\end{example}
\begin{example} 
\textbf{Example:} English set \\
~\\
$S$ as the set of English words, $<$ is the dictionary order (by letters)
\begin{itemize}
	\item "a" $<$ "aa" $<$ "ab" $<$ "b"
\end{itemize}
\end{example}
Notation:
\begin{itemize}
	\item $x<y$ and $y>x$ are the same
	\item $x\le y$ means $x<y$ or $x=y$
\end{itemize}
\begin{definition} 
\textbf{Definition:} Bounded above \\
~\\
Let $S$ be an ordered set and $E$ be a subset of $S$ ($E\subset S$) \\
~\\
We say $E$ is {\color{blue} \textbf{bounded above}} if there exists an element $\theta\in S: \forall x\in E: x\le\theta$ \\ 
~\\
$\theta$ is an {\color{blue} \textbf{upper bound(ub)}} of $E$
\begin{itemize}
	\item Note: This requires a "universal bounding set" or else the idea of being bounded above doesn't make sense
\end{itemize}
\end{definition}
\begin{example} 
\textbf{Example:}  \\
~\\
Let $S = \Q$ with the standard ordering. Let $E=A$ (from the last section, $0<a$, $a^2<2$)
\begin{itemize}
	\item Then $E$ is bounded above by $p=2$ (or really any element $p>\sqrt{2}$) 
\end{itemize}
For $p\in E$, check $2-p$
\begin{align}
	2-p &= \frac{4-p^2}{(2+p)} \\
	    &> \frac{4-2}{2+p} \\
	    &>0
\end{align}
\end{example}
\begin{example} 
\textbf{Example:}  \\
~\\
Let $S=A$ and $E=A$ (not a proper subset), in this case $E$ is not bounded above
\begin{itemize}
	\item We know that for any element $\theta\in E$ there exists another element in $E$ that is larger (from last lecture)
\end{itemize}
\end{example}
\begin{definition} 
\textbf{Definition:} Bounded below \\
~\\
Let $S$ be an ordered set, $E\subset S$. $E$ is {\color{blue} \textbf{bounded below}} if $\exists \beta\in S: \forall x\in E: \beta \le  x$
\begin{itemize}
	\item $\beta$ is called a {\color{blue} \textbf{lower bound (lb)}}
\end{itemize}
\end{definition}
\subsection{Least upper bounds and greatest lower bounds}
\begin{definition} 
	\textbf{Definition:} Least upper bound (LUB)\\
~\\
Let $S$ be an ordered set, subset $E$ bounded above. If $\exists\alpha\in S$ such that 
\begin{itemize}
	\item $\alpha$ is a UB for $E$
	\item If $\gamma < \alpha$, then $\gamma$ is NOT a UB for $E$
\end{itemize}
Then $\alpha$ is called the {\color{blue} \textbf{least upper bound (LUB)}} or {\color{blue} \textbf{supremum}}
\begin{itemize}
	\item Notation: $\alpha = sup(E)$
\end{itemize}
\end{definition}
\begin{remark} 
\textbf{Remark:} Why can we say that $\alpha$ is THE supremum? How do we know that it's unique?
\end{remark}
\begin{definition} 
	\textbf{Definition:} Greatest lower bound (GLB) \\
~\\
Similarly the {\color{blue} \textbf{greatest lower bound}} or {\color{blue} \textbf{infenum}} is the element $\alpha$ (if it exists) st 
\begin{itemize}
	\item $\alpha$ is a LB for $E$
	\item $\gamma > \alpha$ then gamma is not a LB for $E$
	\item Notation: $\alpha = inf(E)$
\end{itemize}
\end{definition}
\begin{example} 
\textbf{Example:}  \\
~\\
Let $S=\Q$ with the normal ordering. $E = \left\{\frac{1}{n}: n\in\N\right\}$
\begin{itemize}
	\item What is the supremum? $sup(E) = 1$
	\item What is the infenum? $inf(E) = 0$
\end{itemize}
Things to check:
\begin{itemize}
	\item Are they rational (in the universal set $S$)? Well yes
	\item Are they a UB/LB? Yes
	\item The hard part: Prove that they are the greatest/least lower/upper bound (todo)
\end{itemize}
Note:
\begin{itemize}
	\item $E$ contains its supremum, but not its infenum
\end{itemize}
\end{example}
\begin{definition} 
	\textbf{Definition:} Least upper bound property (LUB property) \\
~\\
An ordered set $S$ has the LUB property if $\forall E\subset S$ where $E$ is not the empty set, and $E$ is bounded above, then $E$ has a least upper bound (in $S$)
\begin{itemize}
	\item All subsets of $S$ that are bounded above, have a LUB
\end{itemize}
There is also a parallel definition for the {\color{blue} \textbf{greatest upper bound property}}.
\end{definition}
\begin{example} 
\textbf{Example:}  \\
~\\
Does $\Z$ have the LUB property? \\
~\\
What about $\Q$? \\
\begin{itemize}
	\item How would you go about proving it?
	\begin{itemize}
		\item Not having the statement is more straightforward because you can just find a counter-example
		\item Considering arbitrary subsets is more difficult
	\end{itemize}
\end{itemize}
\begin{enumerate}
	\item I think the answers are yes and no.
\end{enumerate}
\end{example}
\begin{theorem} 
\textbf{Theorem:} {\color{blue} } \\
~\\
(Rudin 1.11) Let $S$ be an ordered set.
\begin{itemize}
	\item $S$ has the LUB property $\iff$ $S$ has the GLB property
\end{itemize}
\end{theorem}
\begin{proof} 
\textbf{Proof:} \\
\hrule
\bigskip
Forward: \\
~\\
Let $S$ be an ordered set with the LUB property. (WTS $S$ has the GLB property). ~\\
~\\
Let $E\subset S$ with $E$ nonempty and bounded below (assumptions for the GLB property) (wts $E$ has an infenum) \\
~\\
Let $L = \{x\in S: $ x is a LB for $E \}$. $L$ is non-empty because $E$ is bounded below \\
~\\
If $y\in E$, then $y$ is an UB for $L$. (Because all elements of $L$ are less than all elements of $E$). Since $E$ is non-empty, $L$ is bounded above. ~\\
~\\
So now we have set $L\subset S$ that is non-empty and bounded above, because $S$ has the LUB property, $L$ has supremum $\alpha$. Claim that this $\alpha$ is the infenum of $E$. \\
~\\
$\alpha \le x : \forall x\in E$, hence $\alpha$ is a lower bound for $E$, so $\alpha$ is an element of $L$ (why?) \\
\begin{itemize}
	\item $\forall \gamma\in S$, $\gamma < \alpha$ implies $\gamma$ is not an upper bound of $L$. 
	\item Since all values in $E$ are upper bounds of $L$, $\gamma\notin E$.
	\item So since being less than $\alpha$ implies it is not in $E$, it follows that $\forall x\in E: \alpha \le x$
\end{itemize}
~\\
Since $\alpha$ is the supremum of $L$, $\alpha \ge \gamma: \forall \gamma\in L$. Since $L$ is the set of all upper bounds of $E$, we get that $\alpha$ is the greatest lower bound of $E$ \\
\qed
\bigskip
\hrule
\bigskip
Backward: (exercise, should be very similar)
\end{proof}
\begin{remark} 
\textbf{Remark:} General proof notes:
\begin{itemize}
	\item Use the structure and values that we have access to to generate the desired values. Ie use facts about bounded above/below, use the LUB property to get a solid value in $S$, the supremum of $L$. 
	\item The fact that we know so little about $S$ and $E$ makes things easier, because there only are so many things you can try to create values from
\end{itemize}
\end{remark}
\end{document}
