\documentclass{article}
\usepackage[utf8]{inputenc}
\usepackage{amsfonts}
\usepackage{amsmath}
\usepackage{amssymb}
\usepackage{framed}
\usepackage[framemethod=tikz]{mdframed}
\usepackage{braket}

\newcommand{\Z}{\mathbb{Z}}
\newcommand{\R}{\mathbb{R}}
\newcommand{\C}{\mathbb{C}}
\newcommand{\Q}{\mathbb{Q}}
\newcommand{\N}{\mathbb{N}}

\newcommand{\qed}{\null\nobreak\hfill\ensuremath{\square}}

\definecolor{mycolor}{rgb}{0.122, 0.435, 0.698}
\definecolor{exampleBorder}{rgb}{0.8, 0.58, 0.46}
\definecolor{propBorder}{rgb}{0.0, 0.5, 0.0}
\definecolor{theoremBorder}{rgb}{0.2, 0.2, 0.6}
\definecolor{proofBorder}{rgb}{1.0, 0.44, 0.37}
\definecolor{rmkBorder}{rgb}{0.75, 0.58, 0.89}
\definecolor{claimBorder}{rgb}{0.0, 0.42, 0.24}
\definecolor{lemmaBorder}{rgb}{0.4, 0.6, 0.8}

\newmdenv[nobreak=true,innerlinewidth=0.5pt, roundcorner=4pt,linecolor=mycolor,innerleftmargin=12pt,innerrightmargin=12pt,innertopmargin=12pt,innerbottommargin=12pt]{definition}
\newmdenv[leftmargin=1pt, rightmargin=1pt, nobreak=true,innerlinewidth=0.5pt, roundcorner=4pt,linecolor=exampleBorder,innerleftmargin=12pt,innerrightmargin=12pt,innertopmargin=12pt,innerbottommargin=12pt]{example}
\newmdenv[nobreak=true,innerlinewidth=0.5pt, roundcorner=4pt,linecolor=propBorder,innerleftmargin=12pt,innerrightmargin=12pt,innertopmargin=12pt,innerbottommargin=12pt]{prop}
\newmdenv[nobreak=true,innerlinewidth=0.5pt, roundcorner=4pt,linecolor=theoremBorder,innerleftmargin=12pt,innerrightmargin=12pt,innertopmargin=12pt,innerbottommargin=12pt]{theorem}
\newmdenv[nobreak=true,innerlinewidth=0.5pt, roundcorner=4pt,linecolor=proofBorder,innerleftmargin=12pt,innerrightmargin=12pt,innertopmargin=12pt,innerbottommargin=12pt]{proof}
\newmdenv[nobreak=true,innerlinewidth=0.5pt, roundcorner=4pt,linecolor=rmkBorder,innerleftmargin=12pt,innerrightmargin=12pt,innertopmargin=12pt,innerbottommargin=12pt]{remark}
\newmdenv[nobreak=true,innerlinewidth=0.5pt, roundcorner=4pt,linecolor=claimBorder,innerleftmargin=12pt,innerrightmargin=12pt,innertopmargin=12pt,innerbottommargin=12pt]{claim}
\newmdenv[nobreak=true,innerlinewidth=0.5pt, roundcorner=4pt,linecolor=lemmaBorder,innerleftmargin=12pt,innerrightmargin=12pt,innertopmargin=12pt,innerbottommargin=12pt]{lemma}

\title{Math 320: Chapter 1}
\author{rctcwyvrn}
\date{September 2020}

\begin{document}

\maketitle

\section{Common sets}
\begin{itemize}
	\item Natural numbers $\N$ (Rudin uses $\mathbb{J}$)
		\begin{itemize}
			\item Set notation $\left\{ 1, 2, 3, 4, \ldots\right\}$
			\item Closed under addition and multiplication
			\item Not closed under subtraction (might get a negative number)
		\end{itemize}
	\item Integers $\Z$
		\begin{itemize}
			\item Set notation $\ldots -3, -2, -1, 0,1,2,3,4,\ldots$
			\item Closed under subtraction and addition
			\item But not under division
		\end{itemize}
	\item Rationals $\Q$
		\begin{itemize}
			\item Set notation: $\left\{\frac{m}{n}: m\in\Z, n\in \N \right\}$
			\begin{itemize}
				\item Where $\frac{m_1}{n_1} = \frac{m_2}{n_2}$ iff $m_1*n_2 = m_2*n_1$
				\item Our intuitive idea of division
			\end{itemize}
		\item Alternative we could define $\Q$ in a more straightforward way
			\begin{itemize}
				\item Define $\Q$ as the set of ordered pairs $\left\{(m,n): m\in\Z, n\in \N  \right\}$
				\item We also need to make sure the equivalent fractions are accounted for
				\item where $(m_1,n_1)$ is equivalent to $(m_2,n_2)$ (ie $(m_1,n_1) \sim (m_2,n_2)$) if $m_1n_2 = m_2n_1$
			\end{itemize}
		\item Closed under addition, subtraction, multiplication, and division (if divisor is non-zero)
		\item Question: Are the rationals sufficient for everything we want to do in real analysis (in calculus)?
		\begin{itemize}
			\item Can we do things in calculus? 
			\item Can we take limits? (and have them work properly?)
		\end{itemize}
		\item Nope! (note: the name of this course is \textbf{real} analysis, not \textbf{rational} analysis
		\item The problem is that the rationals have holes, they're not "filled in all the way" like the reals are
	\end{itemize}
\end{itemize}
\begin{example} 
		\textbf{Example:} Holes \\
		~\\
		(Rudin 1.1a)  We want to show that there's a number we can't reach in the rationals ($\sqrt{2}$). $\nexists p\in\Q : p^2=2$  
		\begin{proof} 
		\textbf{Proof:} \\
		~\\
		By contradiction: Suppose that $\exists p\in\Q : p^2=2$. Then since $p$ is rational, we can write $p = \frac{m}{n}$, $m\in\Z$, $n\in\N$.
		\begin{itemize}
			\item WLOG we may suppose that $m$ and $n$ are not both even (because if not then we could just divide both by 2 and have a new $m$ and $n$ and repeat until this statement is true)
			\begin{itemize}
				\item Q: How can we be sure that this dividing by 2 will eventually end? (More explicitly, how many times do we need to do it?)
			\end{itemize}
			\item We have $2 = p^2 = \frac{m^2}{n^2}$
			\begin{align}
				2 &= \frac{m^2}{n^2} \\
				m^2 &= 2n^2
			\end{align}
		\item So $m$ must be even
		\item Ie $m = 2k$ for some odd integer $k$
		\begin{align}
			2n^2 &= m^2 \\
			2n^2 &= (2k)^2 \\
			n^2 &= 2k^2 
		\end{align}
		\item So $n$ must be even, but we said that one of them had to be odd.
		\item Contradiction!
		\end{itemize}
		\end{proof}
	\end{example}
	\begin{example} \textbf{Example:} \\
		\\
		(Rudin 1.1b) 
		\begin{itemize}
			\item Let $A = \left\{ p\in\Q: p>0, p^2<2\right\}$
			\item Let $B = \left\{ p\in\Q : p>0, p^2>2\right\}$
			\item Consider the area where the two sets meet. We know that they meet at $\sqrt{2}$, which is not a rational. 
			\item So we know that the set of rationals has little holes in it, like $\sqrt{2}$ which make us unable to use limits, which are important for real analysis 
		\end{itemize}
		Then:
		\begin{itemize}
			\item $\forall p\in A, \exists q\in A: p<q$ (A does not have a largest element)
			\item Similarly $B$ does not have a smallest element ($\forall p\in B, \exists q\in B: q<p$
		\end{itemize}
		\begin{proof} 
		\textbf{Proof:} \\
		~\\
		First one
		\begin{itemize}
			\item Consider arbitrary $p\in A$. Try $q = \frac{2p+2}{2+p}$ 
			\item Check $q\in\Q$, the denominator is non-zero and rational, so the result is rational.
			\item Check $q>0$, both numerator and denominator are positive
			\item Check $q^2<2$.
			\begin{align}
				q^2 &= \frac{(2p+2)^2}{(2+p)^2} \\
				    &= \frac{2(p^2-2)}{(p+2)^2} +2
			\end{align}
			\item The fraction is less than zero, because the numerator is negative ($p^2 <2$ because $p\in A$), so $q^2<2)$
			\item Check that $p<q$. Well $q = p + \frac{2-p^2}{2+p}$, which is positive, so $q>p$
		\end{itemize}
		Second one
		\begin{itemize}
			\item Exercise (Try the exact same choice of $q$)
		\end{itemize}
		\end{proof}
	\item Q: Where did that $q$ come from? (Think calculus)
	\item Q: Why do we care so much that there is $q^2=2$, but don't care that there isn't a $q^2 = -1$?
\end{example}
\end{document}
