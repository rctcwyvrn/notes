\documentclass{article}
\usepackage[utf8]{inputenc}
\usepackage{amsfonts}
\usepackage{amsmath}
\usepackage{amssymb}
\usepackage{framed}
\usepackage[framemethod=tikz]{mdframed}
\usepackage{braket}

\newcommand{\Z}{\mathbb{Z}}
\newcommand{\R}{\mathbb{R}}
\newcommand{\C}{\mathbb{C}}
\newcommand{\Q}{\mathbb{Q}}
\newcommand{\qed}{\null\nobreak\hfill\ensuremath{\square}}

\definecolor{mycolor}{rgb}{0.122, 0.435, 0.698}
\definecolor{exampleBorder}{rgb}{0.8, 0.58, 0.46}
\definecolor{propBorder}{rgb}{0.0, 0.5, 0.0}
\definecolor{theoremBorder}{rgb}{0.2, 0.2, 0.6}
\definecolor{proofBorder}{rgb}{1.0, 0.44, 0.37}
\definecolor{rmkBorder}{rgb}{0.75, 0.58, 0.89}
\definecolor{claimBorder}{rgb}{0.0, 0.42, 0.24}
\definecolor{lemmaBorder}{rgb}{0.4, 0.6, 0.8}

\newmdenv[nobreak=true,innerlinewidth=0.5pt, roundcorner=4pt,linecolor=mycolor,innerleftmargin=12pt,innerrightmargin=12pt,innertopmargin=12pt,innerbottommargin=12pt]{definition}
\newmdenv[nobreak=true,innerlinewidth=0.5pt, roundcorner=4pt,linecolor=exampleBorder,innerleftmargin=12pt,innerrightmargin=12pt,innertopmargin=12pt,innerbottommargin=12pt]{example}
\newmdenv[nobreak=true,innerlinewidth=0.5pt, roundcorner=4pt,linecolor=propBorder,innerleftmargin=12pt,innerrightmargin=12pt,innertopmargin=12pt,innerbottommargin=12pt]{prop}
\newmdenv[nobreak=true,innerlinewidth=0.5pt, roundcorner=4pt,linecolor=theoremBorder,innerleftmargin=12pt,innerrightmargin=12pt,innertopmargin=12pt,innerbottommargin=12pt]{theorem}
\newmdenv[nobreak=true,innerlinewidth=0.5pt, roundcorner=4pt,linecolor=proofBorder,innerleftmargin=12pt,innerrightmargin=12pt,innertopmargin=12pt,innerbottommargin=12pt]{proof}
\newmdenv[nobreak=true,innerlinewidth=0.5pt, roundcorner=4pt,linecolor=rmkBorder,innerleftmargin=12pt,innerrightmargin=12pt,innertopmargin=12pt,innerbottommargin=12pt]{remark}
\newmdenv[nobreak=true,innerlinewidth=0.5pt, roundcorner=4pt,linecolor=claimBorder,innerleftmargin=12pt,innerrightmargin=12pt,innertopmargin=12pt,innerbottommargin=12pt]{claim}
\newmdenv[nobreak=true,innerlinewidth=0.5pt, roundcorner=4pt,linecolor=lemmaBorder,innerleftmargin=12pt,innerrightmargin=12pt,innertopmargin=12pt,innerbottommargin=12pt]{lemma}

\title{Homomorphisms and quotient groups}
\author{rctcwyvrn}
\date{August 2020}

\begin{document}

\maketitle


\section{Homomorphisms and quotient groups}
\subsection{Generators and group presentations}
We can imagine the subgroup generated by $x$ as x being thrown in a box with itself and shook around. So what if we throw in more elements to be shook together? 
\begin{definition} 
\textbf{Definition:} Subsets as group generators \\
~\\
The subgroup generated by a subset S of G, $\braket{S}$, is the set of finite productive between elements of $S$ and their inverses. 
\begin{itemize}
	\item $S = [a,b]$, then $\braket{S}$ is stuff like $abababa$, $a^5b^3ab^2$, $a^{-1}bab^{-100}$
	\item If $\braket{S}=G$, then S is a {\color{blue} \textbf{set of generators}} for $G$
	\item Notation: $\Z= \braket{1}$ 
	\item What if we know that $x$ has a special condition? Notation $\Z/{100\Z} = \braket{x | x^{100} = 1}$ 
	\item Basically that the group can be generated by any element that has the given property
\end{itemize}
\end{definition}
\begin{example} 
\textbf{Example:} $\Z$ \\
~\\
$\braket{1} = \Z$, because each integer can be written as a bunch of $1$'s or a bunch of $-1$'s
\end{example}
\newpage
\begin{definition} 
\textbf{Definition:} Group presentation \\
~\\
We can define a group by a set of generators and {\color{blue} \textbf{relations}} between them. The {\color{blue} \textbf{group presentation}} is that expression 
\begin{itemize}
	\item We can then say that two elements of a group are equal iff you can get from one to the other with the relations.
\end{itemize}
\end{definition}
\begin{example} 
\textbf{Example:} Dihedral group \\
~\\
The group presentation is  \[
 D_{2n} = \braket{r,s | r^n = s^2 = 1, rs=sr^{-1}}
.\]
\begin{itemize}
	\item This defines what the group is by defining the relationships between the generators
\end{itemize}
\end{example}
\begin{example} 
\textbf{Example:} Free group \\
~\\
The {\color{blue} \textbf{free group on n elements}} is the group with $n$ generators and no relations.
\[
F_n = \braket{x_1,x_2,x_3\ldots}
.\] 
\begin{itemize}
	\item Can basically be thought of as arbitrary units being thrown together
	\item $F_2 = \braket{a,b}$ is a bunch of $a$, $b$, $a^{-1}$, $b^{-1}$ thrown together.
	\item $F_1 = \Z$. Why? Because $\Z$ is just the group made up by adding $1$ and $-1$ to itself a bunch of times
\end{itemize}
\end{example}
\begin{remark} 
\textbf{Remark:} The same group can have very different presentations, because a generator values can encompass two or more values of another generator set 
\end{remark}
\newpage
\subsection{Homomorphisms}
How can we define relationships between groups that aren't just isomorphisms?
\begin{definition} 
\textbf{Definition:}  Homomorphism\\
~\\
For groups $(G,\star)$ and $(H,*)$, A {\color{blue} \textbf{group homomorphism}} is a map $\phi: G\to H$ where $\forall g_1, g_2 \in G$ we have
\[
	\phi(g_1\star g_2) = \phi(g_1) * \phi(g_2)
.\] 
\begin{itemize}
	\item Like a linear map, but over groups instead of vector spaces
	\item Note the lack of bijection condition, we only need that the group action is respected
\end{itemize}
\end{definition}
\begin{remark} 
\textbf{Remark:} The right way to think about an isomorphism is as a "bijective homomorphism" 
\end{remark}
\begin{example} 
\textbf{Example:} Homomorphisms
\begin{itemize}
	\item All isomorphisms are homomorphisms
	\item The identity map is a homomorphism
	\item The {\color{blue} \textbf{trivial homomorphism}} sends everything to $1_H$
	\item From $\Z$ to $\Z/{100\Z}$ where you just mod everything by $100$
	\item From $\Z$ to itself where you just multiply everything by $10$
		\begin{itemize}
			\item This map is injective, but not surjective
		\end{itemize}
	\item From permutations $S_n$ to $S_{n+1}$ where you just keep the $n+1$th position constant.
		\begin{itemize}
			\item Again, injective, but not surjective
		\end{itemize}
\end{itemize}
\end{example}
\begin{remark} 
	\textbf{Remark:} Specifying a homomorphism from $\Z\to G$ is the same as just specifying what the image of $1$ is. Because \[
		\phi(n) = \phi(1) * \phi(1) \ldots = \phi(1)^n
	.\]  
\end{remark}
\begin{remark} 
\textbf{Remark:} The last example shows something important.
\center{To specify a homomorphism $G\to H$, we only have to specify where each generator of $G$ goes. Making sure that the relations are still satisfied (?)}
\end{remark}
\begin{lemma} 
\textbf{Lemma:}
\begin{itemize}
	\item $G\cong H$ iff there exists homomorphisms st $\phi \circ \psi = id_H$ and $\psi \circ \phi = id_G$
	\begin{itemize}
		\item Proof: to do later
	\end{itemize}
        \item Let $\phi$ be a homomorphism, then $\phi(1_G)=1_H$ and $\phi(g^{-1})=\phi(g)^{-1}$
	\begin{itemize}
		\item Proof for the first one: 
			\begin{align}
				\phi(g\star 1_G) &= \phi(g)*\phi(1_G) \\
				\phi(g) &= \phi(g)*\phi(1_G) \\
				1_H &= \phi(1_G)
			\end{align}
	\end{itemize}
\end{itemize}
\end{lemma}
\begin{definition} 
\textbf{Definition:} Kernel \\
~\\
The {\color{blue} \textbf{kernel}} of a homomorphism is the subset of $G$ that sends values to $1_H$
\begin{itemize}
	\item It also happens to be a (not necessarily proper) subgroup of $G$, because $1_G$ is always in the kernel and it is closed
	\item Notation: ker$\phi$
\end{itemize}
\end{definition}
\begin{prop} 
\textbf{Proposition:} Kernel determines injectivity
\center{$\phi$ is injective if and only if ker$\phi = \{1_G\}$}
\end{prop}
\begin{example} 
\textbf{Example:} Kernels
\begin{itemize}
	\item The kernel of an isomorphism is just $1_G$
	\item The kernel of the trivial homomorphism (sending everything to $1_H$ is all of $G$ (duh)
	\item The kernel of the map from $Z$ to the cyclic group of size $100$ is $100\Z$, namely all the integer multiples of $100$. (Because the mod 100 of the map sends all of them to $0$, the identity for the cyclic group
	\item $\phi: \Z\to G$ by $n \mapsto g^n$. The kernel then depends on $g$
		\begin{itemize}
			\item If ord$g = \infty$, then the kernel is just $1$
			\item If ord$g = a$, then the kernel is $a\Z = {\ldots a^{-2}, a^{-1}, 1, a, a^2 \ldots}$
		\end{itemize}
\end{itemize}
\end{example}
\begin{remark} 
\textbf{Remark:}  
The image of a homomorphism forms a subgroup as well
\end{remark}

\subsection{Cosets and modding out}
Here's the idea:
\begin{itemize}
	\item Consider a surjective homomorphism $\phi: G\to Q$ that is not injective (ker$\phi$ is non-trivial), what can we say about it?
	\item Consider a related case, $f: \Z\to \Z/{100\Z}$, the kernel is $100\Z$
		\begin{itemize}
			\item We also then know that $f(x) = f(g + x)$ $\forall g\in$ ker$\phi$
			\item This basically means that $f$ doesn't really care about elements of the subgroup $100Z$, it's \textbf{indifferent}.
			\item Similarly, notice that for $N=100\Z$,
				\begin{align}	
					N &= \{\ldots-200,-100,0,100,200\ldots\} \\
					1 + N &= \{\ldots-199,-99,1,101,201\ldots\}\\
					\ldots \\
					99 + N &= \{\ldots-101,-1,99,199,299\ldots\}
				\end{align}
			\item The image for each of those sets is the same, img$(g + N) = \{g\}$
		\end{itemize}
\end{itemize}
\newpage
\begin{definition} 
\textbf{Definition:} Quotient groups \\
~\\
Let $\phi: G\to Q$ be a surjective homomorphism with kernel $N$ (subgroup of $G$)
\center{We claim that in this case, $Q$ should be thought of as the {\color{blue} \textbf{quotient}} of $G$ by $N$}
\begin{itemize}
	\item Notation: $G/N$ 
\end{itemize}
\end{definition}
\begin{remark} 
\textbf{Remark:} We can think of $Q$ as the group whose elements are represented by the sets, ie for the $\Z/100\Z$ homomorphism, the elements of $Q$ can be thought of as these sets 
	\begin{align}	
		N &= \{\ldots-200,-100,0,100,200\ldots\} \\
		1 + N &= \{\ldots-199,-99,1,101,201\ldots\}\\
		\ldots \\
		99 + N &= \{\ldots-101,-1,99,199,299\ldots\}
	\end{align}
	\begin{itemize}
		\item Note how there are exactly 100 of these sets, just like $Q$ (which is $\Z /100\Z$ remember)
		\item If the homomorphism had been an isomorphism, then each one of those sets would have one value, and there would be exactly as many elements as the cardinality of $G$
	\end{itemize}
\end{remark}
\begin{remark} 
	\textbf{Remark:} We can also define an equivalence relation $\sim _N$ on $G$ where $x \sim _N y$ iff $\phi(x) = \phi(y)$, ie they belong to the same set $a + N$ 
\end{remark}
\begin{definition} 
\textbf{Definition:} Left coset \\
~\\
Let $H$ be any subgroup of $G$. A set of the form $gH$ is called a {\color{blue} \textbf{left coset}} of $H$
\end{definition}
\begin{remark} 
	\textbf{Remark:} $g_1N$ is often equal to $g_2N$ even if $g_1\neq g_2$. ie $g_1=3$ and $g_2=103$ for the $\Z /100\Z$ group 
\end{remark}
\begin{remark} 
	\textbf{Remark:} Given cosets $g_1H$ and $g_2H$, $x \mapsto g_2g_1^{-1}x$ is a bijection from $g_1H \to g_2H$ (Note this means all cosets have the same cardinality) 
\end{remark}
\begin{remark} 
	\textbf{Remark:} \textbf{Elements of the quotient group $Q$ are naturally identified with left cosets of the divisor group, $N$ } 
	\begin{itemize}
		\item This is just a formalization of what was mentioned before, that we can think of the elements of the quotient group $Q$ as those sets (which turned out to be the left cosets of N)
	\end{itemize}
\end{remark}
\begin{definition} 
\textbf{Definition:} Normal groups \\
~\\
A subgroup $N$ of $G$ is {\color{blue} \textbf{normal}} if it is the kernel of some homomorphism.
\begin{itemize}
	\item Notation: $N\trianglelefteq G$
	\item Equivalent definitions of a normal subgroup are
	\begin{itemize}
		\item That the subgroup is closed under conjugation from $G$, ie $gng^{-1}\in N$ $\forall g\in G$ and $n\in N$
		\item That the cosets of $N$ form a group (the quotient group)
	\end{itemize}
\end{itemize}
\end{definition}
\begin{definition} 
\textbf{Definition:} Quotient groups again \\
~\\
Let $N\trianglelefteq G$, then the {\color{blue} \textbf{quotient group}}, denoted $G /N$ (read: "G mod N") is defined as follows
\begin{itemize}
	\item The elements of $G /N$ are left cosets of $N$
	\item Define the product of two cosets as such
	\begin{itemize}
		\item Recall that each coset corresponds to one value in $G$
		\item Take the product of the cosets using those representatives
		\begin{itemize}
			\item Let $C_1 = g_1N$ and $C_2 = g_2N$. 
			\item Then $C_1 \cdot C_2$ should be the coset $g_1g_2H$ (the coset that contains $g_1g_2$)
		\end{itemize}
	\end{itemize}
\item By this definition, $G/ N$ is isomorphic to $Q$ (the old definition of a quotient group)
\end{itemize}
\end{definition}
Some intuition about quotient groups
\begin{itemize}
	\item The way they make the most sense to me is thinking of them as "the result of organizing $G$ by $N$"
	\item Each coset is a set of values that have some property relating to $N$, (namely that they're all in $g_1N$)
	\item It then makes sense to refer to the resulting groupings of the elements as a collective, instead of by individual elements
	\item So the quotient group is just the labels on the groupings of values based on how they behave relative to $N$
	\item In the case of $\Z /100\Z$ the labels are based on their remainder
	\item Note that after we mod out, we don't care about the individual elements but just the labels, the representative elements
	\item The normal condition on the subgroup just ensures that the labels both capture all the values in $G$ and that the cosets form a group themselves
\end{itemize}

\subsection{Proof of Langrange's Theorem}

\begin{theorem} 
\textbf{Theorem:} {\color{blue} Langrange's theorem} \\
~\\
Let $G$ be a finite group, $H$ any subgroup. Then $|H|$ divides $|G|$
\end{theorem}
\begin{proof} 
\textbf{Proof:} \\
~\\
All the cosets of $H$ form a partition (??) of $G$ (though not necessarily a group, since $H$ might not be normal). So if $n$ is the number of cosets, then $n\dot |H|=|G|$ (?? Why does this equal $|G|$ and not just some proportion of it?)
\end{proof}
\begin{remark} 
	\textbf{Remark:} In general, for finite groups $G$ and normal subgroup $H$, $|G /N = |G| /|N|$ 
\end{remark}

\subsection{Eliminating the homomorphism}
Recap: Quotient groups
\begin{itemize}
	\item The elements of the quotient group $G /N$ are cosets $gN$
	\item The group operation is $g_1N \cdot g_2N= (g_1g_2)N$
\end{itemize}
Where do we use the requirement that $N$ is a normal subgroup in the quotient group definition?
~\\
~\\
Answer: We don't know what $g_1$ or $g_2$ are, so we need to guarantee that the group operation's $g_1g_2$ will end up at the same coset, no matter which $g_1$ and $g_2$ are picked from those respective cosets. \textbf{The condition that $N$ must be a normal subgroup gives us this.}
\begin{itemize}
	\item We get this because $N$ is defined as the kernel of some homomorphism$\phi$
	\item Why? I don't really get it... The book says it's because all values in the coset $gN$ have the same value under the homomorphism $\phi$
\end{itemize}
\begin{lemma} 
\textbf{Lemma:} \\
~\\
For $\phi: G \to K$ is a homomorphism with $H=$ker$\phi$. If $h\in H$ and $g\in G$, then $ghg^{-1} \in H$.
\begin{proof} 
\textbf{Proof:} \\
~\\
Show that $\phi(ghg^{-1}) = 1_k$ (show that it's in the kernel, which is $H$).
\begin{align}
	\phi(ghg^{-1}) &= \\ 
		       &= \phi(g) * \phi(h) * \phi(g^{-1}) \\
		       &= \phi(g) * \phi(g^{-1}) \\
		       &= \phi(gg^{-1}) \\
		       &= \phi(1_G) \\
		       &= 1_K \\
\end{align}
\end{proof}
\end{lemma}
Turns out the converse is also true
\begin{theorem} 
\textbf{Theorem:} {\color{blue} Algebraic condition for normal subgroups} \\
~\\
Let $H$ be a subgroup of $G$, then the following are equivalent 
\begin{itemize}
	\item $H \trianglelefteq G$
	\item $\forall g\in G$ and $h\in H$, $ghg^{-1} \in H$
\end{itemize}
\begin{proof} 
\textbf{Proof:} \\
~\\
The last proof showed one direction. For the other one, we need to build a homomorphism with kernel $H$. We can do this by
\begin{itemize}
	\item Defining the quotient group directly as the cosets, and verifying that it is valid
	\item Let our homomorphism be the map from $G$ to our newly built quotient group
\end{itemize}
For the group operation we need
\begin{claim} 
\textbf{Claim:} \\
~\\
If $g_1' \sim _H g_1$ and $g_2' \sim _H g_2$, then $g_1'g_2' \sim _H g_1g_2$
\end{claim}
Short proof:
\begin{itemize}
	\item $g_1' = g_1h_1$ and $g_2'=g_2h_2$ because they're in the same respective cosets. 
	\item $H$ has the property that $g_2^{-1}h_1g_2$ is some element of $H$, $h_3$, then multiply both sides on the left by $g_2$, leaving $h_1g_2 = g_2h_3$
	\item So $g_1(h_1g_2)h_2 = g_1(g_2h_3)h_2 = g_1g_2(h_3h_2)$, which is $\sim _H g_1g_2$ because the $h_3h_2 \in H$  
\end{itemize}
This means that our group operation is consistent, no matter which element in the coset we pick, so now define the group operation to be
\[
	(g_1H)*(g_2H) = (g_1g_2)H
.\] 
So now we have our $G /H$ group, with a projection map $g \mapsto gH$, which has a kernel $H$
\begin{itemize}
	\item Elements of $H$ get sent to $hH$, which is just $H$, which is the identity element
\end{itemize}
\qed
\end{proof}
\end{theorem}
\begin{example} 
\textbf{Example:} Modding out a product group \\
~\\
Earlier we had the trivial subgroup of $G \times H$,
\[
	G' = \{(g,1_H) | g\in G\}
.\]
We can show that
\begin{itemize}
	\item $G' \trianglelefteq G \times H$
	\begin{itemize}
		\item Just need to show the $ghg^{-1} \in G'$ condition forall $g\in G\times H$ and $h\in G'$
	\end{itemize}
	\item Also that $(G\times H) /G' \cong H$
	\begin{itemize}
		\item Intuition: The different cosets are entirely dependent on the $h$ value in the $(g,h)$ pairs that represent the cosets. ie each $h$ gets it's own coset, with all the values of $g$
		\item $(g,h) \sim _{G'} (1_G,h)$ for any $g,h$
	\end{itemize}
\end{itemize}
\end{example}
\begin{remark} 
\textbf{Remark:} Suppose $G$ is abelian. Then all subgroups of $G$ are normal.
\begin{proof} 
\textbf{Proof:} \\
~\\
It's really easy to show the $ghg^{-1} \in H$ condition because we can just reorder it to $gg^{-1}h = 1_Gh = h$, which is in $H$ by definition
\end{proof}
\end{remark}
\end{document}

