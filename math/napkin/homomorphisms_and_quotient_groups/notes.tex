\documentclass{article}
\usepackage[utf8]{inputenc}
\usepackage{amsfonts}
\usepackage{amsmath}
\usepackage{amssymb}
\usepackage{framed}
\usepackage[framemethod=tikz]{mdframed}
\usepackage{braket}

\newcommand{\Z}{\mathbb{Z}}
\newcommand{\R}{\mathbb{R}}
\newcommand{\C}{\mathbb{C}}
\newcommand{\Q}{\mathbb{Q}}

\definecolor{mycolor}{rgb}{0.122, 0.435, 0.698}
\definecolor{exampleBorder}{rgb}{0.8, 0.58, 0.46}
\definecolor{propBorder}{rgb}{0.0, 0.5, 0.0}
\definecolor{theoremBorder}{rgb}{0.2, 0.2, 0.6}
\definecolor{proofBorder}{rgb}{1.0, 0.44, 0.37}
\definecolor{rmkBorder}{rgb}{0.75, 0.58, 0.89}
\definecolor{claimBorder}{rgb}{0.0, 0.42, 0.24}
\definecolor{lemmaBorder}{rgb}{0.4, 0.6, 0.8}

\newmdenv[innerlinewidth=0.5pt, roundcorner=4pt,linecolor=mycolor,innerleftmargin=12pt,innerrightmargin=12pt,innertopmargin=12pt,innerbottommargin=12pt]{definition}
\newmdenv[innerlinewidth=0.5pt, roundcorner=4pt,linecolor=exampleBorder,innerleftmargin=12pt,innerrightmargin=12pt,innertopmargin=12pt,innerbottommargin=12pt]{example}
\newmdenv[innerlinewidth=0.5pt, roundcorner=4pt,linecolor=propBorder,innerleftmargin=12pt,innerrightmargin=12pt,innertopmargin=12pt,innerbottommargin=12pt]{prop}
\newmdenv[innerlinewidth=0.5pt, roundcorner=4pt,linecolor=theoremBorder,innerleftmargin=12pt,innerrightmargin=12pt,innertopmargin=12pt,innerbottommargin=12pt]{theorem}
\newmdenv[innerlinewidth=0.5pt, roundcorner=4pt,linecolor=proofBorder,innerleftmargin=12pt,innerrightmargin=12pt,innertopmargin=12pt,innerbottommargin=12pt]{proof}
\newmdenv[innerlinewidth=0.5pt, roundcorner=4pt,linecolor=rmkBorder,innerleftmargin=12pt,innerrightmargin=12pt,innertopmargin=12pt,innerbottommargin=12pt]{remark}
\newmdenv[innerlinewidth=0.5pt, roundcorner=4pt,linecolor=claimBorder,innerleftmargin=12pt,innerrightmargin=12pt,innertopmargin=12pt,innerbottommargin=12pt]{claim}
\newmdenv[innerlinewidth=0.5pt, roundcorner=4pt,linecolor=lemmaBorder,innerleftmargin=12pt,innerrightmargin=12pt,innertopmargin=12pt,innerbottommargin=12pt]{lemma}

\title{Homomorphisms and quotient groups}
\author{rctcwyvrn}
\date{August 2020}

\begin{document}

\maketitle


\section{Homomorphisms and quotient groups}
\subsection{Generators and group presentations}
We can imagine the subgroup generated by $x$ as x being thrown in a box with itself and shook around. So what if we throw in more elements to be shook together? 
\begin{definition} 
\textbf{Definition:} Subsets as group generators \\
~\\
The subgroup generated by a subset S of G, $\braket{S}$, is the set of finite productive between elements of $S$ and their inverses. 
\begin{itemize}
	\item $S = [a,b]$, then $\braket{S}$ is stuff like $abababa$, $a^5b^3ab^2$, $a^{-1}bab^{-100}$
	\item If $\braket{S}=G$, then S is a {\color{blue} \textbf{set of generators}} for $G$
	\item Notation: $\Z= \braket{1}$ 
	\item What if we know that $x$ has a special condition? Notation $\Z/{100\Z} = \braket{x | x^{100} = 1}$ 
	\item Basically that the group can be generated by any element that has the given property
\end{itemize}
\end{definition}
\begin{example} 
\textbf{Example:} $\Z$ \\
~\\
$\braket{1} = \Z$, because each integer can be written as a bunch of $1$'s or a bunch of $-1$'s
\end{example}
\newpage
\begin{definition} 
\textbf{Definition:} Group presentation \\
~\\
We can define a group by a set of generators and {\color{blue} \textbf{relations}} between them. The {\color{blue} \textbf{group presentation}} is that expression 
\begin{itemize}
	\item We can then say that two elements of a group are equal iff you can get from one to the other with the relations.
\end{itemize}
\end{definition}
\begin{example} 
\textbf{Example:} Dihedral group \\
~\\
The group presentation is  \[
 D_{2n} = \braket{r,s | r^n = s^2 = 1, rs=sr^{-1}}
.\]
\begin{itemize}
	\item This defines what the group is by defining the relationships between the generators
\end{itemize}
\end{example}
\begin{example} 
\textbf{Example:} Free group \\
~\\
The {\color{blue} \textbf{free group on n elements}} is the group with $n$ generators and no relations.
\[
F_n = \braket{x_1,x_2,x_3\ldots}
.\] 
\begin{itemize}
	\item Can basically be thought of as arbitrary units being thrown together
	\item $F_2 = \braket{a,b}$ is a bunch of $a$, $b$, $a^{-1}$, $b^{-1}$ thrown together.
	\item $F_1 = \Z$. Why? Because $\Z$ is just the group made up by adding $1$ and $-1$ to itself a bunch of times
\end{itemize}
\end{example}
\begin{remark} 
\textbf{Remark:} The same group can have very different presentations, because a generator values can encompass two or more values of another generator set 
\end{remark}
\newpage
\subsection{Homomorphisms}
How can we define relationships between groups that aren't just isomorphisms?
\begin{definition} 
\textbf{Definition:}  Homomorphism\\
~\\
For groups $(G,\star)$ and $(H,*)$, A {\color{blue} \textbf{group homomorphism}} is a map $\phi: G\to H$ where $\forall g_1, g_2 \in G$ we have
\[
	\phi(g_1\star g_2) = \phi(g_1) * \phi(g_2)
.\] 
\begin{itemize}
	\item Like a linear map, but over groups instead of vector spaces
	\item Note the lack of bijection condition, we only need that the group action is respected
\end{itemize}
\end{definition}
\begin{remark} 
\textbf{Remark:} The right way to think about an isomorphism is as a "bijective homomorphism" 
\end{remark}
\begin{example} 
\textbf{Example:} Homomorphisms
\begin{itemize}
	\item All isomorphisms are homomorphisms
	\item The identity map is a homomorphism
	\item The {\color{blue} \textbf{trivial homomorphism}} sends everything to $1_H$
	\item From $\Z$ to $\Z/{100\Z}$ where you just mod everything by $100$
	\item From $\Z$ to itself where you just multiply everything by $10$
		\begin{itemize}
			\item This map is injective, but not surjective
		\end{itemize}
	\item From permutations $S_n$ to $S_{n+1}$ where you just keep the $n+1$th position constant.
		\begin{itemize}
			\item Again, injective, but not surjective
		\end{itemize}
\end{itemize}
\end{example}
\begin{remark} 
	\textbf{Remark:} Specifying a homomorphism from $\Z\to G$ is the same as just specifying what the image of $1$ is. Because \[
		\phi(n) = \phi(1) * \phi(1) \ldots = \phi(1)^n
	.\]  
\end{remark}
\begin{remark} 
\textbf{Remark:} The last example shows something important.
\center{To specify a homomorphism $G\to H$, we only have to specify where each generator of $G$ goes. Making sure that the relations are still satisfied (?)}
\end{remark}
\begin{lemma} 
\textbf{Lemma:}
\begin{itemize}
	\item $G\cong H$ iff there exists homomorphisms st $\phi \circ \psi = id_H$ and $\psi \circ \phi = id_G$
	\begin{itemize}
		\item Proof: to do later
	\end{itemize}
        \item Let $\phi$ be a homomorphism, then $\phi(1_G)=1_H$ and $\phi(g^{-1})=\phi(g)^{-1}$
	\begin{itemize}
		\item Proof for the first one: 
			\begin{align}
				\phi(g\star 1_G) &= \phi(g)*\phi(1_G) \\
				\phi(g) &= \phi(g)*\phi(1_G) \\
				1_H &= \phi(1_G)
			\end{align}
	\end{itemize}
\end{itemize}
\end{lemma}
\begin{definition} 
\textbf{Definition:} Kernel \\
~\\
The {\color{blue} \textbf{kernel}} of a homomorphism is the subset of $G$ that sends values to $1_H$
\begin{itemize}
	\item It also happens to be a (not necessarily proper) subgroup of $G$, because $1_G$ is always in the kernel and it is closed
	\item Notation: ker$\phi$
\end{itemize}
\end{definition}
\begin{prop} 
\textbf{Proposition:} Kernel determines injectivity
\center{$\phi$ is injective if and only if ker$\phi = \{1_G\}$}
\end{prop}
\begin{example} 
\textbf{Example:} Kernels
\begin{itemize}
	\item The kernel of an isomorphism is just $1_G$
	\item The kernel of the trivial homomorphism (sending everything to $1_H$ is all of $G$ (duh)
	\item The kernel of the map from $Z$ to the cyclic group of size $100$ is $100\Z$, namely all the integer multiples of $100$. (Because the mod 100 of the map sends all of them to $0$, the identity for the cyclic group
	\item $\phi: \Z\to G$ by $n \mapsto g^n$. The kernel then depends on $g$
		\begin{itemize}
			\item If ord$g = \infty$, then the kernel is just $1$
			\item If ord$g = a$, then the kernel is $a\Z = {\ldots a^{-2}, a^{-1}, 1, a, a^2 \ldots}$
		\end{itemize}
\end{itemize}
\end{example}
\begin{remark} 
\textbf{Remark:}  
The image of a homomorphism forms a subgroup as well
\end{remark}

\subsection{Cosets and modding out}
Here's the idea:
\begin{itemize}
	\item Consider a surjective homomorphism $\phi: G\to Q$ that is not injective (ker$\phi$ is non-trivial), what can we say about it?
	\item Consider a related case, $f: \Z\to \Z/{100\Z}$, the kernel is $100\Z$
		\begin{itemize}
			\item We also then know that $f(x) = f(g + x)$ $\forall g\in$ ker$\phi$
			\item This basically means that $f$ doesn't really care about elements of the subgroup $100Z$, it's \textbf{indifferent}.
			\item Similarly, notice that for $N=100\Z$,
				\begin{align}	
					N &= \{\ldots-200,-100,0,100,200\ldots\} \\
					1 + N &= \{\ldots-199,-99,1,101,201\ldots\}\\
					\ldots \\
					99 + N &= \{\ldots-101,-1,99,199,299\ldots\}
				\end{align}
			\item The image for each of those sets is the same, img$(g + N) = \{g\}$
		\end{itemize}
\end{itemize}
\newpage
\begin{definition} 
\textbf{Definition:} Quotient groups \\
~\\
Let $\phi: G\to Q$ be a surjective homomorphism with kernel $N$ (subgroup of $G$)
\center{We claim that in this case, $Q$ should be thought of as the {\color{blue} \textbf{quotient}} of $G$ by $N$}
\begin{itemize}
	\item Notation: $G/N$ 
\end{itemize}
\end{definition}
\begin{remark} 
\textbf{Remark:} We can think of $Q$ as the group whose elements are represented by the sets, ie for the $\Z/100\Z$ homomorphism, the elements of $Q$ can be thought of as these sets 
	\begin{align}	
		N &= \{\ldots-200,-100,0,100,200\ldots\} \\
		1 + N &= \{\ldots-199,-99,1,101,201\ldots\}\\
		\ldots \\
		99 + N &= \{\ldots-101,-1,99,199,299\ldots\}
	\end{align}
	\begin{itemize}
		\item Note how there are exactly 100 of these sets, just like $Q$ (which is $\Z /100\Z$ remember)
		\item If the homomorphism had been an isomorphism, then each one of those sets would have one value, and there would be exactly as many elements as the cardinality of $G$
	\end{itemize}
\end{remark}
\begin{remark} 
	\textbf{Remark:} We can also define an equivalence relation $\sim _N$ on $G$ where $x \sim _N y$ iff $\phi(x) = \phi(y)$, ie they belong to the same set $a + N$ 
\end{remark}
\begin{definition} 
\textbf{Definition:} Left coset \\
~\\
Let $H$ be any subgroup of $G$. A set of the form $gH$ is called a {\color{blue} \textbf{left coset}} of $H$
\end{definition}
\begin{remark} 
	\textbf{Remark:} $g_1N$ is often equal to $g_2N$ even if $g_1\neq g_2$. ie $g_1=3$ and $g_2=103$ for the $\Z /100\Z$ group 
\end{remark}
\begin{remark} 
	\textbf{Remark:} Given cosets $g_1H$ and $g_2H$, $x \mapsto g_2g_1^{-1}x$ is a bijection from $g_1H \to g_2H$ (Note this means all cosets have the same cardinality) 
\end{remark}
\begin{remark} 
	\textbf{Remark:} \textbf{Elements of the quotient group $Q$ are naturally identified with left cosets of the divisor group, $N$ } 
	\begin{itemize}
		\item This is just a formalization of what was mentioned before, that we can think of the elements of the quotient group $Q$ as those sets (which turned out to be the left cosets of N)
	\end{itemize}
\end{remark}
\begin{definition} 
\textbf{Definition:} Normal groups \\
~\\
A subgroup $N$ of $G$ is {\color{blue} \textbf{normal}} if it is the kernel of some homomorphism.
\begin{itemize}
	\item Notation: $N\trianglelefteq G$
\end{itemize}
\end{definition}
\begin{definition} 
\textbf{Definition:} Quotient groups again \\
~\\
Let $N\trianglelefteq G$, then the {\color{blue} \textbf{quotient group}}, denoted $G /N$ (read: "G mod N") is defined as follows
\begin{itemize}
	\item The elements of $G /N$ are left cosets of $N$
	\item Define the product of two cosets as such
	\begin{itemize}
		\item Recall that each coset corresponds to one value in $Q$ (what the fuck does $Q$ refer to here? The quotient group? The ONE THAT SUPPOSED TO BE MADE UP OF COSETS? I AM CONFUSION)
		\item Take the product of cosets to be the coset corresponding to the product of the values in $Q$ for the two cosets
		\item Note: We can do this with the representatives of the cosets
		\begin{itemize}
			\item Let $C_1 = g_1N$ and $C_2 = g_2N$. 
			\item Then $C_1 \cdot C_2$ should be the coset $g_1g_2H$ (the coset that contains $g_1g_2$)
		\end{itemize}
	\end{itemize}
\item By this definition, $G/ N$ is isomorphic to $Q$ (the old definition of a quotient group)
\end{itemize}
\end{definition}
\end{document}
