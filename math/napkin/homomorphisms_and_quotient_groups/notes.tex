\documentclass{article}
\usepackage[utf8]{inputenc}
\usepackage{amsfonts}
\usepackage{amsmath}
\usepackage{framed}
\usepackage[framemethod=tikz]{mdframed}
\usepackage{braket}

\newcommand{\Z}{\mathbb{Z}}
\newcommand{\R}{\mathbb{R}}
\newcommand{\C}{\mathbb{C}}
\newcommand{\Q}{\mathbb{Q}}

\definecolor{mycolor}{rgb}{0.122, 0.435, 0.698}
\definecolor{exampleBorder}{rgb}{0.8, 0.58, 0.46}
\definecolor{propBorder}{rgb}{0.0, 0.5, 0.0}
\definecolor{theoremBorder}{rgb}{0.2, 0.2, 0.6}
\definecolor{proofBorder}{rgb}{1.0, 0.44, 0.37}
\definecolor{rmkBorder}{rgb}{0.75, 0.58, 0.89}
\definecolor{claimBorder}{rgb}{0.0, 0.42, 0.24}
\definecolor{lemmaBorder}{rgb}{0.4, 0.6, 0.8}

\newmdenv[innerlinewidth=0.5pt, roundcorner=4pt,linecolor=mycolor,innerleftmargin=12pt,innerrightmargin=12pt,innertopmargin=12pt,innerbottommargin=12pt]{definition}
\newmdenv[innerlinewidth=0.5pt, roundcorner=4pt,linecolor=exampleBorder,innerleftmargin=12pt,innerrightmargin=12pt,innertopmargin=12pt,innerbottommargin=12pt]{example}
\newmdenv[innerlinewidth=0.5pt, roundcorner=4pt,linecolor=propBorder,innerleftmargin=12pt,innerrightmargin=12pt,innertopmargin=12pt,innerbottommargin=12pt]{prop}
\newmdenv[innerlinewidth=0.5pt, roundcorner=4pt,linecolor=theoremBorder,innerleftmargin=12pt,innerrightmargin=12pt,innertopmargin=12pt,innerbottommargin=12pt]{theorem}
\newmdenv[innerlinewidth=0.5pt, roundcorner=4pt,linecolor=proofBorder,innerleftmargin=12pt,innerrightmargin=12pt,innertopmargin=12pt,innerbottommargin=12pt]{proof}
\newmdenv[innerlinewidth=0.5pt, roundcorner=4pt,linecolor=rmkBorder,innerleftmargin=12pt,innerrightmargin=12pt,innertopmargin=12pt,innerbottommargin=12pt]{remark}
\newmdenv[innerlinewidth=0.5pt, roundcorner=4pt,linecolor=claimBorder,innerleftmargin=12pt,innerrightmargin=12pt,innertopmargin=12pt,innerbottommargin=12pt]{claim}
\newmdenv[innerlinewidth=0.5pt, roundcorner=4pt,linecolor=lemmaBorder,innerleftmargin=12pt,innerrightmargin=12pt,innertopmargin=12pt,innerbottommargin=12pt]{lemma}

\title{Homomorphisms and quotient groups}
\author{rctcwyvrn}
\date{August 2020}

\begin{document}

\maketitle


\section{Homomorphisms and quotient groups}
\subsection{Generators and group presentations}
We can imagine the subgroup generated by $x$ as x being thrown in a box with itself and shook around. So what if we throw in more elements to be shook together? 
\begin{definition} 
\textbf{Definition:} Subsets as group generators \\
~\\
The subgroup generated by a subset S of G, $\braket{S}$, is the set of finite productive between elements of $S$ and their inverses. 
\begin{itemize}
	\item $S = [a,b]$, then $\braket{S}$ is stuff like $abababa$, $a^5b^3ab^2$, $a^{-1}bab^{-100}$
	\item If $\braket{S}=G$, then S is a {\color{blue} \textbf{set of generators}} for $G$
	\item Notation: $\Z= \braket{1}$ 
	\item What if we know that $x$ has a special condition? Notation $\Z/{100\Z} = \braket{x | x^{100} = 1}$ 
	\item Basically that the group can be generated by any element that has the given property
\end{itemize}
\end{definition}
\begin{example} 
\textbf{Example:} $\Z$ \\
~\\
$\braket{1} = \Z$, because each integer can be written as a bunch of $1$'s or a bunch of $-1$'s
\end{example}
\newpage
\begin{definition} 
\textbf{Definition:} Group presentation \\
~\\
We can define a group by a set of generators and {\color{blue} \textbf{relations}} between them. The {\color{blue} \textbf{group presentation}} is that expression 
\begin{itemize}
	\item We can then say that two elements of a group are equal iff you can get from one to the other with the relations.
\end{itemize}
\end{definition}
\begin{example} 
\textbf{Example:} Dihedral group \\
~\\
The group presentation is  \[
 D_{2n} = \braket{r,s | r^n = s^2 = 1, rs=sr^{-1}}
.\]
\begin{itemize}
	\item This defines what the group is by defining the relationships between the generators
\end{itemize}
\end{example}
\begin{example} 
\textbf{Example:} Free group \\
~\\
The {\color{blue} \textbf{free group on n elements}} is the group with $n$ generators and no relations.
\[
F_n = \braket{x_1,x_2,x_3\ldots}
.\] 
\begin{itemize}
	\item Can basically be thought of as arbitrary units being thrown together
	\item $F_2 = \braket{a,b}$ is a bunch of $a$, $b$, $a^{-1}$, $b^{-1}$ thrown together.
	\item $F_1 = \Z$. Why? Because $\Z$ is just the group made up by adding $1$ and $-1$ to itself a bunch of times
\end{itemize}
\end{example}
\begin{remark} 
\textbf{Remark:} The same group can have very different presentations, because a generator values can encompass two or more values of another generator set 
\end{remark}
\newpage
\subsection{Homomorphisms}
How can we define relationships between groups that aren't just isomorphisms?
\begin{definition} 
\textbf{Definition:}  Homomorphism\\
~\\
For groups $(G,\star)$ and $(H,*)$, A {\color{blue} \textbf{group homomorphism}} is a map $\phi: G\to H$ where $\forall g_1, g_2 \in G$ we have
\[
	\phi(g_1\star g_2) = \phi(g_1) * \phi(g_2)
.\] 
\begin{itemize}
	\item Like a linear map, but over groups instead of vector spaces
	\item Note the lack of bijection condition, we only need that the group action is respected
\end{itemize}
\end{definition}
\begin{remark} 
\textbf{Remark:} The right way to think about an isomorphism is as a "bijective homomorphism" 
\end{remark}
\begin{example} 
\textbf{Example:} Homomorphisms
\begin{itemize}
	\item All isomorphisms are homomorphisms
	\item The identity map is a homomorphism
	\item The {\color{blue} \textbf{trivial homomorphism}} sends everything to $1_H$
	\item From $\Z$ to $\Z/{100\Z}$ where you just mod everything by $100$
	\item From $\Z$ to itself where you just multiply everything by $10$
		\begin{itemize}
			\item This map is injective, but not surjective
		\end{itemize}
	\item From permutations $S_n$ to $S_{n+1}$ where you just keep the $n+1$th position constant.
		\begin{itemize}
			\item Again, injective, but not surjective
		\end{itemize}
\end{itemize}
\end{example}
\begin{remark} 
	\textbf{Remark:} Specifying a homomorphism from $\Z\to G$ is the same as just specifying what the image of $1$ is. Because \[
		\phi(n) = \phi(1) * \phi(1) \ldots = \phi(1)^n
	.\]  
\end{remark}
\begin{remark} 
\textbf{Remark:} The last example shows something important.
\center{To specify a homomorphism $G\to H$, we only have to specify where each generator of $G$ goes. Making sure that the relations are still satisfied (?)}
\end{remark}
\begin{lemma} 
\textbf{Lemma:}
\begin{itemize}
	\item $G\cong H$ iff there exists homomorphisms st $\phi \circ \psi = id_H$ and $\psi \circ \phi = id_G$
	\begin{itemize}
		\item Proof: to do later
	\end{itemize}
        \item Let $\phi$ be a homomorphism, then $\phi(1_G)=1_H$ and $\phi(g^{-1})=\phi(g)^{-1}$
	\begin{itemize}
		\item Proof for the first one: 
			\begin{align}
				\phi(g\star 1_G) &= \phi(g)*\phi(1_G) \\
				\phi(g) &= \phi(g)*\phi(1_G) \\
				1_H &= \phi(1_G)
			\end{align}
	\end{itemize}
\end{itemize}
\end{lemma}
\begin{definition} 
\textbf{Definition:} Kernel \\
~\\
The {\color{blue} \textbf{kernel}} of a homomorphism is the subset of $G$ that sends values to $1_H$
\begin{itemize}
	\item It also happens to be a (not necessarily proper) subgroup of $G$, because $1_G$ is always in the kernel and it is closed
	\item Notation: ker$\phi$
\end{itemize}
\end{definition}
\begin{prop} 
\textbf{Proposition:} Kernel determines injectivity
\center{$\phi$ is injective if and only if ker$\phi = \{1_G\}$}
\end{prop}
\begin{example} 
\textbf{Example:} Kernels
\begin{itemize}
	\item The kernel of an isomorphism is just $1_G$
	\item The kernel of the trivial homomorphism (sending everything to $1_H$ is all of $G$ (duh)
	\item The kernel of the map from $Z$ to the cyclic group of size $100$ is $100\Z$, namely all the integer multiples of $100$. (Because the mod 100 of the map sends all of them to $0$, the identity for the cyclic group
	\item $\phi: \Z\to G$ by $n \mapsto g^n$. The kernel then depends on $g$
		\begin{itemize}
			\item If ord$g = \infty$, then the kernel is just $1$
			\item If ord$g = a$, then the kernel is $a\Z = {\ldots a^{-2}, a^{-1}, 1, a, a^2 \ldots}$
		\end{itemize}
\end{itemize}
\end{example}
\begin{remark} 
\textbf{Remark:}  
The image of a homomorphism forms a subgroup as well
\end{remark}

\subsection{Cosets and modding out}
\end{document}
