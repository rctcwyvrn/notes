\documentclass{article}
\usepackage[utf8]{inputenc}
\usepackage{amsfonts}
\usepackage{amsmath}
\usepackage{amssymb}
\usepackage{framed}
\usepackage[framemethod=tikz]{mdframed}
\usepackage{braket}

\newcommand{\Z}{\mathbb{Z}}
\newcommand{\R}{\mathbb{R}}
\newcommand{\C}{\mathbb{C}}
\newcommand{\Q}{\mathbb{Q}}
\newcommand{\N}{\mathbb{N}}
\newcommand{\qed}{\null\nobreak\hfill\ensuremath{\square}}

\definecolor{defBorder}{rgb}{0.122, 0.435, 0.698}
\definecolor{exampleBorder}{rgb}{0.8, 0.58, 0.46}
\definecolor{propBorder}{rgb}{0.0, 0.5, 0.0}
\definecolor{theoremBorder}{rgb}{0.2, 0.2, 0.6}
\definecolor{proofBorder}{rgb}{1.0, 0.44, 0.37}
\definecolor{rmkBorder}{rgb}{0.75, 0.58, 0.89}
\definecolor{claimBorder}{rgb}{0.0, 0.42, 0.24}
\definecolor{lemmaBorder}{rgb}{0.4, 0.6, 0.8}

\newmdenv[nobreak=true,innerlinewidth=0.5pt, roundcorner=4pt,linecolor=defBorder,innerleftmargin=12pt,innerrightmargin=12pt,innertopmargin=12pt,innerbottommargin=12pt]{definition}
\newmdenv[nobreak=true,innerlinewidth=0.5pt, roundcorner=4pt,linecolor=exampleBorder,innerleftmargin=12pt,innerrightmargin=12pt,innertopmargin=12pt,innerbottommargin=12pt]{example}
\newmdenv[nobreak=true,innerlinewidth=0.5pt, roundcorner=4pt,linecolor=propBorder,innerleftmargin=12pt,innerrightmargin=12pt,innertopmargin=12pt,innerbottommargin=12pt]{prop}
\newmdenv[nobreak=true,innerlinewidth=0.5pt, roundcorner=4pt,linecolor=theoremBorder,innerleftmargin=12pt,innerrightmargin=12pt,innertopmargin=12pt,innerbottommargin=12pt]{theorem}
\newmdenv[nobreak=true,innerlinewidth=0.5pt, roundcorner=4pt,linecolor=proofBorder,innerleftmargin=12pt,innerrightmargin=12pt,innertopmargin=12pt,innerbottommargin=12pt]{proof}
\newmdenv[nobreak=true,innerlinewidth=0.5pt, roundcorner=4pt,linecolor=rmkBorder,innerleftmargin=12pt,innerrightmargin=12pt,innertopmargin=12pt,innerbottommargin=12pt]{remark}
\newmdenv[nobreak=true,innerlinewidth=0.5pt, roundcorner=4pt,linecolor=claimBorder,innerleftmargin=12pt,innerrightmargin=12pt,innertopmargin=12pt,innerbottommargin=12pt]{claim}
\newmdenv[nobreak=true,innerlinewidth=0.5pt, roundcorner=4pt,linecolor=lemmaBorder,innerleftmargin=12pt,innerrightmargin=12pt,innertopmargin=12pt,innerbottommargin=12pt]{lemma}

\title{Math 322}
\author{rctcwyvrn}
\date{August 2020}
\begin{document}
\maketitle

\section{Lecture 1}
\begin{definition} 
\textbf{Definition:} Partition \\
~\\
Let $X$ be a set, $X_i$ subsets such that each $x\in X$ is in exactly one $X_i$ (they make a {\color{blue} \textbf{disjoint union of $X$}})
\begin{itemize}
	\item The $X_i$ are then a {\color{blue} \textbf{partition}} of X
	\item Note: The $X_i$ maybe of differing sizes
	\item There may also be an infinite number of them
\end{itemize}
\end{definition}
\begin{definition} 
\textbf{Definition:} Equivalence relation \\
~\\
Elements of $X$ are uniquely grouped by which $X_i$ they're in, so we can use this to define an {\color{blue} \textbf{equivalence relation ($\sim$)}} ie: $\alpha\sim\beta$ if they live in the same $X_i$ \\
~\\
Properties
\begin{itemize}
	\item Reflexive: $x\sim x$ always
	\item Transitive: $a\sim b$ and $b\sim c$ means $a\sim c$
	\item Symmetric: $x\sim y$ implies $y\sim x$
\end{itemize}
\end{definition}
\begin{definition}
\textbf{Definition:} Equivalence classes \\
~\\
Given a partition $\left\{X_i\right\}$ of $X$, you can form a new set whose elements are the sets $X_0, X_1, X_2 \ldots$. This is called $X/ \sim$. \\
~\\ 
Elements of that set are {\color{blue} \textbf{equivalence classes}}
\begin{itemize}
	\item If $\alpha\in X_i$ then $\alpha$ is called a {\color{blue} \textbf{representative}} of the equivalence class $X_i$
	\item Notation: $X_i = \left[\alpha\right] = \overline{\alpha}$
	\item We can also define a {\color{blue} \textbf{projection}} from $X$ to $X /\sim$, which sends $\alpha$ to its equivalence class
\end{itemize}
\begin{align}
	\pi: X &\to X /\sim \\
           \alpha &\mapsto \left[\alpha\right]
\end{align}
\begin{itemize}
	\item Sends all the elements of a partition into the one equivalence class
\end{itemize}
\end{definition}
\begin{example} 
\textbf{Example:} Odds and evens \\
~\\
Let $X$ be the integers, with the partition $X_1$ evens, $X_2$ odds then $X /\sim$ is the two element set. ~\\
~\\
Note: Even though both the odds and the evens have infinite size, the quotient set has only two elements
\end{example}
\begin{example} 
\textbf{Example:} Integers mod $N$ \\
~\\
Define $x \sim y$ if they have the same remainder mod $N$.
\begin{itemize}
	\item $N$ equivalence classes (one for each remainder, $0\ldots N-1$)
	\item Notation: $\Z /N\Z$ (reason later (quotient groups)
	\item If $N=2$ then it's just the evens and odds example from before
	\item If $N=7$ then there's $7$ elements, \\
		the equivalence classes for $0,1,2,3,4,5,6$
\end{itemize}
\end{example}
Division of integers (in $\Z$) (Some stuff very closely related to math 312, which we aren't going to prove)
\begin{itemize}
	\item Let $a,b$ integers , $a>0$
	\item Then we have $b= qa+r$, $0 <r\le a-1$, $q$ integer
	\item Note: the quotient and remainder are unique
	\item $a|b \iff r=0$ divisible iff the remainder is zero 
	\item We get that any integer $n$ is unique defined by the primes that make it up
\end{itemize}
\begin{lemma} 
\textbf{Lemma:} \\
~\\
If $a,b$ have no common factors, then $\exists m,n \in\Z : ma+mb=1$
\begin{itemize}
	\item Notation: $(a,b)$: greatest common factor
	\item Notation: $\left[a,b\right]$: least common multiple
\end{itemize}
\end{lemma}
\begin{example} 
\textbf{Example:}  \\
~\\
$7,5$ share no common factors, $m=-2$, $n=3$, $-14 + 15 = 1$
\end{example}
\begin{remark} 
\textbf{Remark:}  Division turns out to generate the basic structural properties of $\Z$
\end{remark}
\begin{lemma} 
\textbf{Lemma:} \\
~\\
Suppose now that $a,b$ share a gcd $d$, then $\exists m,n\in\Z: ma+mb=d$. So the earlier lemma is just a special case (when the gcd is 1)
\begin{itemize}
	\item Note: because $d$ divides both $a$ and $b$, then it should be able to divide $ma+nb$ for any $m,n$
\end{itemize}
\end{lemma}
\end{document}
