\documentclass{article}
\usepackage[utf8]{inputenc}
\usepackage{amsfonts}
\usepackage{amsmath}
\usepackage{amssymb}
\usepackage{framed}
\usepackage[framemethod=tikz]{mdframed}
\usepackage{braket}

\newcommand{\Z}{\mathbb{Z}}
\newcommand{\R}{\mathbb{R}}
\newcommand{\C}{\mathbb{C}}
\newcommand{\Q}{\mathbb{Q}}
\newcommand{\N}{\mathbb{N}}
\newcommand{\qed}{\null\nobreak\hfill\ensuremath{\square}}

\definecolor{defBorder}{rgb}{0.122, 0.435, 0.698}
\definecolor{exampleBorder}{rgb}{0.8, 0.58, 0.46}
\definecolor{propBorder}{rgb}{0.0, 0.5, 0.0}
\definecolor{theoremBorder}{rgb}{0.2, 0.2, 0.6}
\definecolor{proofBorder}{rgb}{1.0, 0.44, 0.37}
\definecolor{rmkBorder}{rgb}{0.75, 0.58, 0.89}
\definecolor{claimBorder}{rgb}{0.0, 0.42, 0.24}
\definecolor{lemmaBorder}{rgb}{0.4, 0.6, 0.8}

\newmdenv[nobreak=true,innerlinewidth=0.5pt, roundcorner=4pt,linecolor=defBorder,innerleftmargin=12pt,innerrightmargin=12pt,innertopmargin=12pt,innerbottommargin=12pt]{definition}
\newmdenv[nobreak=false,innerlinewidth=0.5pt, roundcorner=4pt,linecolor=exampleBorder,innerleftmargin=12pt,innerrightmargin=12pt,innertopmargin=12pt,innerbottommargin=12pt]{example}
\newmdenv[nobreak=true,innerlinewidth=0.5pt, roundcorner=4pt,linecolor=propBorder,innerleftmargin=12pt,innerrightmargin=12pt,innertopmargin=12pt,innerbottommargin=12pt]{prop}
\newmdenv[nobreak=true,innerlinewidth=0.5pt, roundcorner=4pt,linecolor=theoremBorder,innerleftmargin=12pt,innerrightmargin=12pt,innertopmargin=12pt,innerbottommargin=12pt]{theorem}
\newmdenv[nobreak=true,innerlinewidth=0.5pt, roundcorner=4pt,linecolor=proofBorder,innerleftmargin=12pt,innerrightmargin=12pt,innertopmargin=12pt,innerbottommargin=12pt]{proof}
\newmdenv[nobreak=true,innerlinewidth=0.5pt, roundcorner=4pt,linecolor=rmkBorder,innerleftmargin=12pt,innerrightmargin=12pt,innertopmargin=12pt,innerbottommargin=12pt]{remark}
\newmdenv[nobreak=true,innerlinewidth=0.5pt, roundcorner=4pt,linecolor=claimBorder,innerleftmargin=12pt,innerrightmargin=12pt,innertopmargin=12pt,innerbottommargin=12pt]{claim}
\newmdenv[nobreak=true,innerlinewidth=0.5pt, roundcorner=4pt,linecolor=lemmaBorder,innerleftmargin=12pt,innerrightmargin=12pt,innertopmargin=12pt,innerbottommargin=12pt]{lemma}

\title{Math 322 Lecture 2}
\author{rctcwyvrn}
\date{Sept 15 2020}
\begin{document}
\maketitle

\section{Quotient sets continued}
The quotient set $\Z /n\Z$ has a kind of addition on it
\begin{align}
	[\alpha] + [\beta] = [\alpha+\beta]
\end{align}
But is it well defined? (is it consistent for different values of $\alpha$ and $\beta$?) \\
~\\
Suppose $\alpha\sim\alpha '$, so $\alpha -\alpha' = kn$, similarly for $\beta$ and $\beta'$. So we can rewrite 
\begin{align}
	\alpha + \beta &= \alpha' + \beta' + (kn+jn)
\end{align}
Which implies that $\alpha' + \beta' \sim \alpha + \beta$, so the addition is well defined. \\
~\\
We can add and subtract in this set, so it's a group
\begin{definition} 
\textbf{Definition:} Group \\
~\\
A {\color{blue} \textbf{group}} is a set $G$ with a binary operation "multiplication" $(\cdot)$ such that
\begin{itemize}
	\item Associative $(g_1\cdot g_2) \cdot g_3 = g_1\cdot (g_2\cdot g_3)$
	\item Identity $\exists 1_g: \forall x\in G: 1_g \cdot x = x\cdot 1_g = x$
	\item Inverse: $\forall x\in G: \exists y\in G: x\cdot y = y\cdot x = 1_g$
\end{itemize}
A group is a {\color{blue} \textbf{commutative group}} or {\color{blue} \textbf{abelian group}} if
\begin{itemize}
	\item $\forall x,y \in G: x\cdot y= y\cdot x$
	\item Notation: use $+$ as the operator for a commutative group
\end{itemize}
\end{definition}
\begin{example} 
\textbf{Example:} Obvious examples \\
~\\
$\Z, +$
\end{example}
\begin{example} 
\textbf{Example:}  \\
~\\
$(\Z /N\Z, +)$ is a group
\begin{itemize}
	\item It is also finite and commutative
\end{itemize}
\end{example}
\begin{example} 
\textbf{Example:} Matricies \\
~\\
$n$ x $n$ matricies are a group under $+$. The identity matrix is all zero, and the inverse matrix $m$ is $-m$
\begin{itemize}
	\item What about under multiplication?
	\item No, because not all matricies have inverses (so you need non-zero determinant matricies)
	\begin{itemize}
		\item So the set of $n$ by $n$ matricies with non-zero determinant is a group
		\item It is also non-commutative (matrix multiplication order matters)
	\end{itemize}
\end{itemize}
\begin{remark} 
	\textbf{Remark:} We didn't actually check associativity (it's usually just trivial or annoying) 
\end{remark}
\end{example}
\begin{remark} 
\textbf{Remark:} Composition of functions is associative
\begin{itemize}
	\item ie $x\to y\to z\to w$
	\item $h\circ (g\circ f) = (h\circ g)\circ f$
	\item We can use this to check associativity easily
\end{itemize}
\end{remark}
\begin{example} 
\textbf{Example:}  \\
~\\
Let $A$ be a set, and $S(A)$ the set of all bijective maps $f: A\to A$. 
\begin{itemize}
	\item $(S(A),\circ)$ is a group
	\item The symmetric group on the set $A$
	\item You can also thing of each of these functions as a permutation of the order of the elements
\end{itemize}
Check
\begin{itemize}
	\item Associativity: $\circ$ is associative
	\item Identity: The identity function
	\item Inverse: The inverse function (which is why we require bijection)
\end{itemize}
Properties (of $S_n = S(\{1,2,3,\ldots\}$)
\begin{itemize}
	\item Highly non-commutative
	\item $n!$ elements
\end{itemize}
\end{example}
So a group is just a set that we can multiply and divide (multiply by inverse) that satisfies certain rules (depending on the group)
\subsection{Cycle decomposition of permutations}
Consider the element of $S_7$
\begin{align}
	1\to 3 \\
	3\to 5 \\
	\ldots
\end{align}
Is there a better way to write it?
\begin{align}
	(1\to 3\to 5\to \ldots) 
\end{align}
This is called the {\color{blue} \textbf{cycle representation}} (usually written without the arrows)
\begin{example} 
\textbf{Example:}  \\
~\\
$(135)(467)(2)$ each cycle is grouped together, the entire function is 
\begin{itemize}
	\item $1\to 3\to 5\to 1$
	\item $4\to 6\to 7\to 4$
	\item $2\to 2$
\end{itemize} 
Note:
\begin{itemize}
	\item We see that $\sigma ^m = 1$ for some $m$, namely the LCM of the cycle lengths
	\item Note that each cycle has separate values, so they are {\color{blue} \textbf{disjoint}}
\end{itemize}
What about non disjoint cycles? Like $(1,2,3)(3,5,7)$
\begin{align}
	\sigma &= \sigma _1 \circ \sigma _2 \\
	\sigma_1 &= (123) \\ 
	\sigma_2 &= (357) \\
	\intertext{The group operator is composition, so compose them}
	\intertext{Note the order matters, $\sigma_2(\sigma_1)$ is}
	\sigma &= (12573)(4)(6) \\
	\intertext{The other order is $\sigma_2$ first then $\sigma_1$, which is the convention for this class( $\sigma_1\circ\sigma_2 = \sigma_1(\sigma_2(x))$}
	\sigma &= (12357)(4)(6)
	\intertext{So in $S_7$}
	(123)(357) &= (12357)
\end{align}
Note: $(123)$ is really just shorthand for $(123)(4)(5)(6)(7)$ in $S_7$
\begin{itemize}
	\item They're called {\color{blue} \textbf{fixed points}}
	\item Note: it does depend on which set you're in ( $(1,2,3)$ means different things in $S_7$ and $S_{100}$)
\end{itemize}
\begin{remark} 
\textbf{Remark:}  Not all cycles are disjoint, but you can rewrite a {\color{blue} \textbf{disjoint decomposition}} for non disjoint cycles, like $(123)(357)$
\end{remark}
\begin{remark} 
	\textbf{Remark:} Disjoint cycles will always commute with each other, because they don't share any elements. ie $(1,2,3)(4,5,6) = (4,5,6)(1,2,3)$ 
	\begin{itemize}
		\item Note: Not all commutative elements are disjoint though
	\end{itemize}
\end{remark}
\end{example}
\begin{definition} 
\textbf{Definition:} Order \\
~\\
For group $G$, $g\in G$, the {\color{blue} \textbf{order of g}} is the smallest positive integer $m$ such that $g^m = g\cdot g\cdot g\cdot g \ldots = 1_G$
\begin{itemize}
	\item If no such $m$ exists, then $g$ has infinite order
	\item $(1354)(6(2)$ has order $4$
	\item $(123)(45)$ has order $6$, because the LCM (the first time they both end up at the start at the same time) is $6$
\end{itemize}
\end{definition}
\begin{definition} 
\textbf{Definition:} Group order \\
~\\
The order of a group is the number of elements in $G$ (may be infinite)
\end{definition}
\begin{lemma} 
\textbf{Lemma:} \\
~\\
Let $G$ be a finite group, then every element of $G$ has finite order
\end{lemma}
\begin{proof} 
\textbf{Proof:} \\
~\\
Consider $g,g^2,g^3\ldots$. \\
~\\
Since $G$ is finite $\exists r,s: r\neq s: g^r = g^s$ (by pigeonhole, since $G$ is finite it must repeat at some point). \\
$\implies g^r g^{-r} = g^s g^{-r}$ \\
$\implies 1_g = g^{s-r}$ and $s\neq r$, so $g$ has order $s-r$
\end{proof}
General principle: Commutative groups are simpler and easier to make sense of. Non-commutative groups have weird things and hare harder to deal with \\
~\\
Corresponding reading:
\begin{itemize}
	\item 1.1, 1.3
	\item 1.2 next lecture
\end{itemize}
\end{document}
