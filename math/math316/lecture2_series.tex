\documentclass{article}
\usepackage[utf8]{inputenc}
\usepackage{amsfonts}
\usepackage{amsmath}
\usepackage{amssymb}
\usepackage{framed}
\usepackage[framemethod=tikz]{mdframed}
\usepackage{braket}

\newcommand{\Z}{\mathbb{Z}}
\newcommand{\R}{\mathbb{R}}
\newcommand{\C}{\mathbb{C}}
\newcommand{\Q}{\mathbb{Q}}
\newcommand{\N}{\mathbb{N}}
\newcommand{\qed}{\null\nobreak\hfill\ensuremath{\square}}

\definecolor{defBorder}{rgb}{0.122, 0.435, 0.698}
\definecolor{exampleBorder}{rgb}{0.8, 0.58, 0.46}
\definecolor{propBorder}{rgb}{0.0, 0.5, 0.0}
\definecolor{theoremBorder}{rgb}{0.2, 0.2, 0.6}
\definecolor{proofBorder}{rgb}{1.0, 0.44, 0.37}
\definecolor{rmkBorder}{rgb}{0.75, 0.58, 0.89}
\definecolor{claimBorder}{rgb}{0.0, 0.42, 0.24}
\definecolor{lemmaBorder}{rgb}{0.4, 0.6, 0.8}

\newmdenv[nobreak=true,innerlinewidth=0.5pt, roundcorner=4pt,linecolor=defBorder,innerleftmargin=12pt,innerrightmargin=12pt,innertopmargin=12pt,innerbottommargin=12pt]{definition}
\newmdenv[nobreak=true,innerlinewidth=0.5pt, roundcorner=4pt,linecolor=exampleBorder,innerleftmargin=12pt,innerrightmargin=12pt,innertopmargin=12pt,innerbottommargin=12pt]{example}
\newmdenv[nobreak=true,innerlinewidth=0.5pt, roundcorner=4pt,linecolor=propBorder,innerleftmargin=12pt,innerrightmargin=12pt,innertopmargin=12pt,innerbottommargin=12pt]{prop}
\newmdenv[nobreak=true,innerlinewidth=0.5pt, roundcorner=4pt,linecolor=theoremBorder,innerleftmargin=12pt,innerrightmargin=12pt,innertopmargin=12pt,innerbottommargin=12pt]{theorem}
\newmdenv[nobreak=true,innerlinewidth=0.5pt, roundcorner=4pt,linecolor=proofBorder,innerleftmargin=12pt,innerrightmargin=12pt,innertopmargin=12pt,innerbottommargin=12pt]{proof}
\newmdenv[nobreak=true,innerlinewidth=0.5pt, roundcorner=4pt,linecolor=rmkBorder,innerleftmargin=12pt,innerrightmargin=12pt,innertopmargin=12pt,innerbottommargin=12pt]{remark}
\newmdenv[nobreak=true,innerlinewidth=0.5pt, roundcorner=4pt,linecolor=claimBorder,innerleftmargin=12pt,innerrightmargin=12pt,innertopmargin=12pt,innerbottommargin=12pt]{claim}
\newmdenv[nobreak=true,innerlinewidth=0.5pt, roundcorner=4pt,linecolor=lemmaBorder,innerleftmargin=12pt,innerrightmargin=12pt,innertopmargin=12pt,innerbottommargin=12pt]{lemma}

\title{Taylor series}
\author{rctcwyvrn}
\date{August 2020}
\begin{document}
\maketitle

\section{Taylor series (review)}
A power series is
\begin{align}
	S(x) = a_0 + a_1x + a_2x^2 \ldots = \sum_{n=0}^{\infty} a_nx^n
\end{align}
Say we know all the derivatives of a function $f(x)$, can we determine a power series? \\
~\\
Let $f(x) = a_0 + a_1x + a_2x^2 \ldots$ and solve for $a_i$.
\begin{align}
	f'(x) &= a_1 + 2a_2x + 3a_3x^2 \ldots \\
	f''(x) &= 2a_2 + 6a_3x \ldots  \\
	f^{n}(x) &= n!a_n + (n+1) \ldots 2 * a_{n+1}
\end{align}
Sub in $x=0$ to determine all the coefficients $a_i$
\begin{align}
	f'(0) &= a_1 \\ 
	f''(0) &= 2a_2 \\
	f^{n}(0) &= n!a_n 
\end{align}
So filling in the coefficients we get the {\color{blue} \textbf{Taylor series}}
\[
	f(x) = f(0) + \frac{f'(0)}{1!}x \ldots \frac{f^{n}(0)}{n!}x^n \ldots
.\] 
In sum notation
\[
	f(x) = \sum_{n=0}^{\infty} \frac{f^{n}(0)}{n!}x^n 
.\]
\begin{definition} 
\textbf{Definition:} Hyperbolic trig \\
~\\
\begin{align}
	\sinh &= \frac{e^x - e^{-x}}{2} = x + \frac{x^3}{3!} + \frac{x^5}{5!} \ldots \\
	\cosh &= \frac{e^x + e^{-x}}{2} = 1 + \frac{x^2}{2!} + \frac{x^4}{4!} \ldots 
\end{align}
The series are just the $\sin$ and $\cos$ ones, but with no negative signs
\begin{itemize}
	\item See notebook for graphical representation of $\sinh$ and $\cosh$
\end{itemize}
\end{definition}
Now back to ODE stuff
\begin{example} 
\textbf{Example:} Generating the derivatives of the solution from the ODE \\
~\\
Given $y'=2y$, $y(0) = 4)$, we know that $y'' = 2y' = 4y$, continuing on we get that $y^{(n)} = 2^n y$ \\
~\\
Now write out the taylor series
\begin{align}
	y(x) &= \sum_{n=0}^{\infty} \frac{y^n(0)}{n!}x^n \\
	     &= \sum_{n=0}^{\infty} \frac{2^n y(0)}{n!}x^n  \\
	     &= 4\sum_{n=0}^{\infty} \frac{2^n}{n!}x^n \\
	     &= 4e^{2x}
\end{align}
And we can check that this is correct by just solving the ODE (which is separable)
\end{example}
But is there a more general way of doing this (getting a Taylor series from a DE)?
\begin{example} 
\textbf{Example:} Undetermined coefficients \\
~\\
Assume that there is a Taylor series, and figure out the coefficients from the DE. \\
\[
Ly = y' - 2y = 0
.\]  
Assume $y$ has a power series $y=\sum_{n=0}^{\infty} a_n x^n$
\begin{align}
	y' &= \sum_{n=1}^{\infty} a_n n x^{n-1}
\end{align}
Now substitute them into $Ly$
\begin{align}
	Ly &= y' - 2y = 0 \\
	   &= 1a_1x^0 + 2a_2x + 3a_3+x^2 \ldots - 2(a_0+ a_1x + a_2x^2 \ldots) \\
	   &= (a_1-2a_0)x^0 + (2a_2-2a_1)x + (3a_3-2a_2)x^2 + \ldots = 0
\end{align}
This should hold for all values of $x$, so each of those coefficients should be zero.
\begin{itemize}
	\item Alternatively we can think of $x$, $x^2$ \ldots as linearly independent values, and the sum is equal to zero, which implies that the coefficients are zero
\end{itemize}
\begin{align}
	a_1 &= 2a_0 \\
	a_2 &= \frac{2a_1}{2} = 2^2 \frac{a_0}{2} \\
	a_3 &= \frac{2a_2}{3} = 2^3 \frac{a_0}{3!} \\
	\ldots \\
	a_n &= 2^n \frac{a_0}{n!} \\
\end{align}
So we now have
\begin{align}
	y(x) &= a_0 (1 + 2x + \ldots \frac{2^n}{n!}x^n \\
	     &= a_0 e^{2x}
\end{align}
\end{example}
Try it again, this time with just sum notation instead
\begin{example} 
\textbf{Example:}  \\
~\\
$y = \sum_{n=0}^{\infty} a_n x^n$, so $y' = \sum_{n=1}^{\infty} a_n n x^{n-1}$ \\
\begin{align}
		Ly &= y' - 2y = 0 \\
		   &= \sum_{n=1}^{\infty} a_n n x^{n-1} - 2\sum_{n=0}^{\infty} a_n x^n \\
		   &= \sum_{m=0}^{\infty} a_{m+1} (m+1) x^m - 2\sum_{m=0}^{\infty} a_m x^m \\
		   &=  \sum_{m=0}^{\infty} (a_{m+1} (m+1) - 2a_m) x^m \\ 
\end{align}
By the same argument as before, all these coefficients should be $0$, 
\[
	a_{m+1} = \frac{2a_m}{m+1}
.\] 
which gives us a recursive formula for the coefficients. Start at $m=0$
\begin{align}
	a_1 &= \frac{2a_0}{1} \\
	a_2 &= \frac{2a_1}{2} = \frac{2^2 a_0}{2} \\
	a_3 &= \frac{2^3a_0}{3!} \\
	\ldots
	a_n &= \frac{2^n a_0}{n!}
\end{align}
And then we end up at the same solution, $y(x) = a_0 e^{2x}$
\end{example}
Now for a more complex example
\begin{example} 
\textbf{Example:} Ainy Equation \\
\[
Ly = y'' - xy = 0
.\] 
Let $y =\sum_{n=0}^{\infty} a_n x^n$, $y'' = \sum_{n=2}^{\infty} a_n n(n-1) x^{n-2}$
\begin{align}
	Ly &= y'' -xy = 0 \\
	   &= \sum_{n=2}^{\infty} a_n n(n-1) x^{n-2} - \sum_{n=0}^{\infty} a_n x^{n+1} \\ 
	   \intertext{Since we have the n+1 on the right summation, we want to shift the left sum to match, so set it to $m+1 = n-2$}
	   &= a_2(2)(1) + \sum_{m=0}^{\infty} a_{m+3} (m+3)(m+2)x^{m+1} - a_m x^{m+1} \\
	   &= 2a_2 + \sum_{m=0}^{\infty} ( a_{m+3} (m+3)(m+2) - a_m)x^{m+1}
\end{align}
So from that we can solve the coefficients:
\begin{align}
	a_2 &= 0 \\
	a_{m+3} &= \frac{a_m}{(m+3)(m+2)} \\	
	m = 0: a_3 &= \frac{a_0}{3*2} \\
	m = 3: a_6 &= \frac{a_3}{6*5} = \frac{a_0}{6*5*3*2} \\
	m = 1: a_4 &= \frac{a_1}{4*3} \\
	m = 4: a_7 &= \frac{a_4}{7*6} = \frac{a_1}{7*6*4*3}\\
	m = 2: a_5 &= 0 \\
	m = 5: a_8 &= 0
\end{align}
The general solution is the two groups of coefficients, namely
\[
	y(x) = a_0(1 + \frac{1}{3*2}x^3 + \frac{1}{6*5*3*2}x^5 \ldots) + a_1(1 + \frac{1}{4*3}x^4 + \frac{1}{7*6*4*3}x^7 \ldots) 
.\] 
\end{example}
\end{document}
