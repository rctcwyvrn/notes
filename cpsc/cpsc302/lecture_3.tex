\documentclass{article}
\usepackage[utf8]{inputenc}
\usepackage{amsfonts}
\usepackage{amsmath}
\usepackage{amssymb}
\usepackage{framed}
\usepackage[framemethod=tikz]{mdframed}
\usepackage{braket}
\usepackage{listings}

\newcommand{\Z}{\mathbb{Z}}
\newcommand{\R}{\mathbb{R}}
\newcommand{\C}{\mathbb{C}}
\newcommand{\Q}{\mathbb{Q}}
\newcommand{\N}{\mathbb{N}}
\newcommand{\qed}{\null\nobreak\hfill\ensuremath{\square}}

\definecolor{defBorder}{rgb}{0.122, 0.435, 0.698}
\definecolor{exampleBorder}{rgb}{0.8, 0.58, 0.46}
\definecolor{propBorder}{rgb}{0.0, 0.5, 0.0}
\definecolor{theoremBorder}{rgb}{0.2, 0.2, 0.6}
\definecolor{proofBorder}{rgb}{1.0, 0.44, 0.37}
\definecolor{rmkBorder}{rgb}{0.75, 0.58, 0.89}
\definecolor{claimBorder}{rgb}{0.0, 0.42, 0.24}
\definecolor{lemmaBorder}{rgb}{0.4, 0.6, 0.8}

\newmdenv[nobreak=true,innerlinewidth=0.5pt, roundcorner=4pt,linecolor=defBorder,innerleftmargin=12pt,innerrightmargin=12pt,innertopmargin=12pt,innerbottommargin=12pt]{definition}
\newmdenv[nobreak=true,innerlinewidth=0.5pt, roundcorner=4pt,linecolor=exampleBorder,innerleftmargin=12pt,innerrightmargin=12pt,innertopmargin=12pt,innerbottommargin=12pt]{example}
\newmdenv[nobreak=true,innerlinewidth=0.5pt, roundcorner=4pt,linecolor=propBorder,innerleftmargin=12pt,innerrightmargin=12pt,innertopmargin=12pt,innerbottommargin=12pt]{prop}
\newmdenv[nobreak=true,innerlinewidth=0.5pt, roundcorner=4pt,linecolor=theoremBorder,innerleftmargin=12pt,innerrightmargin=12pt,innertopmargin=12pt,innerbottommargin=12pt]{theorem}
\newmdenv[nobreak=true,innerlinewidth=0.5pt, roundcorner=4pt,linecolor=proofBorder,innerleftmargin=12pt,innerrightmargin=12pt,innertopmargin=12pt,innerbottommargin=12pt]{proof}
\newmdenv[nobreak=true,innerlinewidth=0.5pt, roundcorner=4pt,linecolor=rmkBorder,innerleftmargin=12pt,innerrightmargin=12pt,innertopmargin=12pt,innerbottommargin=12pt]{remark}
\newmdenv[nobreak=true,innerlinewidth=0.5pt, roundcorner=4pt,linecolor=claimBorder,innerleftmargin=12pt,innerrightmargin=12pt,innertopmargin=12pt,innerbottommargin=12pt]{claim}
\newmdenv[nobreak=true,innerlinewidth=0.5pt, roundcorner=4pt,linecolor=lemmaBorder,innerleftmargin=12pt,innerrightmargin=12pt,innertopmargin=12pt,innerbottommargin=12pt]{lemma}

\title{CPSC 302}
\author{rctcwyvrn}
\date{Sept 14 2020}
\begin{document}
\maketitle

\section{Pre-class}
\subsection{Numerical algorithms and error}
How to measure errors? (For $v$ approximating $u$) 
\begin{itemize}
	\item Absolute
	\begin{itemize}
		\item The {\color{blue} \textbf{absolute error}} is $|u-v|$
	\end{itemize}
	\item Relative
	\begin{itemize}
		\item The {\color{blue} \textbf{relative error}} is $\frac{|u-v|}{|u|}$
	\end{itemize}
	\item Some combination
\end{itemize}
MATLAB code example
\begin{lstlisting}
	% some examples
	u = [1;	   1;    -1.5; 100;  100]
	v = [0.99; 1.01; -1.2; 99.99; 99]
	abs_error = abs(u-v)
	relative_error = abs(u-v)./abs(u) % use ./ for vector math
\end{lstlisting}
Sources/types of errors
\begin{itemize}
	\item In the model (how close it is to reality)
	\item In the input data
\end{itemize}
Approximation errors
\begin{itemize}
	\item Discritization errors
	\item Convergence errors
	\item Roundoff errors
\end{itemize}
\begin{example} 
\textbf{Example:}  \\
~\\
Given a smooth function $f(x)$. approximate the derivative at $x=x_0$
\begin{itemize}
	\item We get a discritization error when we don't choose a small enough $h$ for calculating the derivative, ie we discretize $\R$ to size $h$
	\item We can then calculate the discritization error from the Taylor series(see notebook)
\end{itemize}
Try with $f(x) = \sin(x)$ at $x_0 = 1.2$. (so we want to approximate $\cos$)
\begin{itemize}
	\item When $h$ is large ($h>10^{-8}$) we see that the error matches the discritization error
	\item For smaller $h$ there is also a roundoff error
	\begin{itemize}
		\item We see that for smaller $h$, around $10^{-10}$ the error begins to grow (not following the error pattern we expected from just discritization)
		\item Why is there roundoff error? Soon tm
	\end{itemize}
\end{itemize}
\end{example}

\section{Lecture}
\begin{itemize}
	\item Q: When is absolute error more important than relative error?
	\item A: Like the height of a flight of stairs, you don't really care how relatively high the stairs are, just that you're not off by a certain amount
\end{itemize}
\begin{itemize}
	\item Q: What about relative error?
	\item A: If you're measuring something that's 100kg, you don't necessarily care if you're off by 1g, but if you're measuring something that weighs 2g, being off by 1g matters alot.
\end{itemize}
Roundoff error
\begin{itemize}
	\item Why does decreasing $h$ not continue to decrease the error? Why does the error start increasing after a certain point?
	\item Idea: As we lower $h$ we increase the proportion that roundoff error affects our calculations involving $h$, which increases the error proportionally in the result
	\item Q: How would you define a roundoff error?
	\item A: Computers don't have infinite memory, so at some point it has to round off the value, which introduces error.
\end{itemize}
Discretization error
\begin{itemize}
	\item When you make finite approximations of continuous processes
\end{itemize}
Convergence errors
\begin{itemize}
	\item Arise from iterative methods
\end{itemize}
\end{document}
